% coding:utf-8
\documentclass[a4paper,10pt,fleqn]{article}

\usepackage[ansinew]{inputenc}
\usepackage{graphics}
\usepackage{graphicx}
\usepackage[english,ngerman]{babel}
\usepackage{longtable}
\usepackage{amsmath}
\usepackage[all]{xy}
\usepackage[xindy]{glossaries}
\usepackage{makeidx}
\usepackage{pdfpages}
\usepackage{graphicx}
\usepackage[pdftex]{hyperref}
\usepackage{printlen}
\usepackage{enumitem}
\usepackage{cite}
\usepackage{url}
\usepackage{pdfpages}                        % Packet für PDF Dateimanipulation laden
\usepackage{fancyhdr}                        % http://en.wikibooks.org/wiki/LaTeX/Page_Layout#Customising_with_fancyhdr

\pagestyle{fancy} %eigener Seitenstil
\fancyhf{} %alle Kopf- und Fußzeilenfelder bereinigen


% Textbreite anpassen
%\addtolength{\textwidth}{1cm}
%\addtolength{\evensidemargin}{-5mm}
%\addtolength{\oddsidemargin}{-5mm}

%\addtolength{\headwidth}{1cm}

% Header Settings
%\rhead{\setlength{\unitlength}{1mm}
%\begin{picture}(-2,7)
%    \includegraphics[width=35mm]{hslu_logo2.PNG}
%\end{picture}}

\fancyhead[L]{MATH1} %Kopfzeile links
\fancyhead[C]{} %zentrierte Kopfzeile
\fancyhead[R]{} %Kopfzeile rechts
%\fancyhead[R]{\includegraphics[scale=0.25]{hslu_logo2.PNG}}
\renewcommand{\headrulewidth}{0.4pt} %obere Trennlinie
\fancyfoot[L]{Daniel Winz}
\fancyfoot[C]{HS - 2012}
\fancyfoot[R]{\thepage}
\renewcommand{\footrulewidth}{0.4pt} %untere Trennlinie

\begin{document}
\tableofcontents
\title{Formelsammlung Mathematik}

\section{Vektoren}
\subsubsection{Vektor zwischen zwei Vektoren}
\begin{equation}
\overrightarrow{P_1 P_2} = 
\left( \begin{array}{c}
x_2 - x_1\\
y_2 - y_1\\
z_2 - z_1
\end{array} \right)
\end{equation}

\subsubsection{Vektorarithmetik}
\begin{flalign}
u + v = v + u\\
0 + u = u + 0 = u\\
k(cu) = kc(u)\\
(k + c)u = ku + cu\\
(u + v) + w = u + (v + w)\\
u + (-u) = 0\\
k(u + v) = ku + kv
\end{flalign}
Skalarprodukt\\
Kreuzprodukt\\

\section{Folgen}
\subsection{arithmetische Folgen}
\subsection{geometrische Folgen}

\section{Reihen}
\subsection{arithmetische Reihen}
\subsection{geometrische Reihen}

\section{Differenzialrechnung}
\subsection{Ableitungsregeln}
Summenregel\\
Produkteregel\\
Quotientenregel\\
Kettenregel\\

\section{Integralrechnung}


\end{document}