% coding:utf-8

%----------------------------------------
%FOSAMATH, a LaTeX-Code for a mathematical summary for basic analysis
%Copyright (C) 2013, Daniel Winz, Ervin Mazlagic, Adrian Imboden, Philipp Langer

%This program is free software; you can redistribute it and/or
%modify it under the terms of the GNU General Public License
%as published by the Free Software Foundation; either version 2
%of the License, or (at your option) any later version.

%This program is distributed in the hope that it will be useful,
%but WITHOUT ANY WARRANTY; without even the implied warranty of
%MERCHANTABILITY or FITNESS FOR A PARTICULAR PURPOSE.  See the
%GNU General Public License for more details.
%----------------------------------------

% coding:utf-8
\section{Reihen}
$S_n$ ist die Summe aller Glieder von $a_1$ bis $a_n$. 
\[ \boxed{S_n = \sum_{k=1}^{n} a_k = a_1 + a_2 + a_3 \cdots + a_n} \]

\subsection{arithmetische Reihe}
\[ \boxed{S_n = \left(n + 1\right)\left(a_0 + d \frac{n}{2}\right) = \left(n + 1\right) \frac{a_0 + a_n}{2}} \]

\subsection{geometrische Reihe}
\[ \boxed{\text{Für } q \neq 1: \quad S_n = a_1 \left(  \frac{q^n - 1}{q - 1} \right) = a_1 \left(  \frac{1 - q^n}{1 - q} \right)} \]

\[ \boxed{\begin{array}{ll}
a_0 :& S_n = a_0 \cdot \dfrac{1-q^{n+1}}{1-q} \\ 
& \\
a_1 :& S_n = a_1 \cdot \dfrac{1-q^n}{1-q}
\end{array}} \]

\[ \boxed{\text{Für } q = 1: \quad S_n = a_1 \left(1+1^1 + 1^2 + \ldots + 1^{n-1}\right) = a_1 n} \]
Unendliche geometrische Reihe
\[ \boxed{\text{Für } |q| < 1: S = \lim_{n \rightarrow \infty} S_n = \frac{a_1}{1 - q}} \] 
%Um das $n$-te Glied einer geometrischen Reihen zu berechnen gilt:\\
%$ q = \frac{a_n}{a_{n-1}} = \frac{a_{n-1}}{a_{n-2}} $ \\
%$ \Rightarrow  a_n = a_{(n-1)} \cdot q = a_{(n-2)} \cdot q^2 = a_{(n - (n-1))} \cdot q^{(n-1)} = a_1 \cdot q^{(n-1)} $
%\[ \boxed{ a_n = a_1 \cdot q^{n-1} } \]
%\[ \boxed{ a_{n+1} = a_1 \cdot q^n } \]
$a_1$ ist hier das erste Element der geometrischen Folge!

\subsection{Konvergenz}

\subsubsection*{Quotientenkriterium}
Um die Konvergenzfrage einer geometrischen Reihe zu klären betrachtet man
\[ \lim\limits_{n \rightarrow \infty} \left| \frac{a_{n+1}}{a_n} \right| = \lim\limits_{n \rightarrow \infty} \left| q \right| \]
Hierbei können drei verschiedene Ergebnisse entstehen:
\[ \boxed{\begin{array}{ll}
|q| < 1 & \text{die Reihe konvergiert} \\
|q| > 1 & \text{die Reihe divergiert} \\
|q| = 1 & \text{keine Aussage möglich}
\end{array}} \]
Dies wird oft als Quotientenkriterium bezeichnet.

\subsubsection*{Wurlzelkrierium}
Eine weitere Methode zur Klärung der Konvergenzfrage liefert das so genannte Wurzelkriterium.
\[ \lim\limits_{n \rightarrow \infty} \left( |a_k| \right)^{\frac{1}{n}} = q\]
Daraus können wie beim Quotientenkriterium drei Fälle eintreten:
\[ \boxed{\begin{array}{ll}
|q| < 1 & \text{die Reihe konvergiert} \\
|q| > 1 & \text{die Reihe divergiert} \\
|q| = 1 & \text{keine Aussage möglich} 
\end{array}} \]


\ifti
\subsection{Reihen mit dem TI89 berechnen}
$\sum$\verb{(Ausdruck,Variable,untere Grenze, obere Grenze){ \\
$\sum$\verb{(EXP,VAR,LOW,HIGH){ \\\\
\begin{tabular}{lll}
EXP  & Ausdruck      & bezeichnet den Term der die Reihe beschreibt \\
VAR  & Variable      & bezeichnet die inkrementierte Variable \\
LOW  & untere Grenze & Anfangspunkt der Inkrementierung \\
HIGH & obere Grenze  & Endpunkt der Inkrementierung \\
\end{tabular}
\fi
\ifnspire
\subsection{Reihen mit dem TI-Nspire berechnen}
\[ \sum_{\boxed{K} ~=~ \boxed{0}}^{\boxed{n}}\boxed{a_n} \]\\
\begin{tabular}{lll}
$\boxed{a_n}$  & Ausdruck      & bezeichnet den Term der die Reihe beschreibt \\
$\boxed{K}$     & Variable      & bezeichnet die inkrementierte Variable \\
$\boxed{0}$    & untere Grenze & Anfangspunkt der Inkrementierung \\
$\boxed{n}$ & obere Grenze  & Endpunkt der Inkrementierung \\
\end{tabular}
\fi
