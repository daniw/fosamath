% coding:utf-8
\section{Vektorgeometrie in der Ebene}

\subsection{Abstand zweier Puntke}
\[ \boxed{ \overline{P_1 P_2} = \sqrt{ (x_2 - x_1)^2 + (y_2 - y_1)^2 } } \]

\subsection{Geradengleichungen}

\subsubsection{Normalform (explizite Form)}
\[ \boxed{ g: y= mx + q }\]
\[ \boxed{ \text{Steigung } m = \frac{y_2 - y_1}{x_2 - x_1} = \frac{\Delta y}{\Delta x}  = tan \varphi } \]

\subsubsection{Koordinatenform (implizite Form)}
\[ \boxed{ g: ax + by + c = 0 } \]

\subsubsection{Achsenabschnittsform}
\[ \boxed{ g: \frac{x}{p} + \frac{y}{q} = 1 } \]

\subsubsection{Hesse'sche Normalform}
\[ \boxed{ g:  \frac{ax + by + c}{\sqrt{ a^2 + b^2 } } = 0 }  \]

\subsubsection{Parameterform}
\[ \boxed{ 
	g: \vec{r} = \vec{r_0} + t \cdot \vec{a}  = 
    \left( 
		\begin{array}{cc} 
	  		x_0 \\ y_0
		\end{array}
	\right)
    + t \cdot 
    \left( 
		\begin{array}{cc} 
			a_x \\ a_y
		\end{array}
    \right)  
   }
\]


\subsection{Normalenvektor}
Der Normalenvektor ist ein Vektor, welcher senkrecht auf einem anderen Vektor bzw. einer Geraden liegt. Hier im Beispiel in welchem $ \vec{n} \bot g(x)$
\[ \boxed{ \vec{n} = 
	\left( 
		\begin{array}{cc} 
	  		n_x \\ n_y
		\end{array}
    \right)
    =
    \left( 
		\begin{array}{cc} 
			a \\ b
		\end{array}
    \right)
    =
    \left( 
		\begin{array}{cc} 
			-a_y \\ a_x
		\end{array}
    \right)
} \]
\noindent
Der Richtungsvektor von $g(x)$ ist 
$  
	\left( 
		\begin{array}{cc} 
	  		a_x \\ a_y
		\end{array}
    \right)
    \Rightarrow 
    \left( 
		\begin{array}{cc} 
	  		-a_y \\ a_x
		\end{array}
    \right)
    = \vec{n}
$.

\subsection{Abstand Punkt zu Gerade}
Für eine Gerade $g: ax + by + c = 0$ und einen Punkt $P_1 (x_1 | y_1)$ gilt:
\[ \boxed{ d = \left| \frac{ax_1 + by_1 + c}{\sqrt{a^2 + b^2}} \right| } \]

\subsection{Schnittwinkel zwischen Geraden}
Für den spitzen Schnittwinkel $\varphi$ zwischen den Geraden 

$g_1: y = m_1x + q_1$ und $g_2: y = m_2x + q_2$ gilt:
\[ \boxed{ tan\varphi = \left| \frac{m_2 - m_1}{1 + m_1 \cdot m_2} \right| } \\ \text{für } \varphi \neq 90^{\circ} \]

\[ \boxed{ g_1 || g_2 \Leftrightarrow m_1 = m_2 \text{und } g_1 \bot g_2 \Leftrightarrow m_2 = - \frac{1}{m_1} } \\ \text{für } m_1 \neq 0 \]

\section{Vektorgeometrie im Raum}

\subsection{Ortsvektor}
Ein Ortsvektor beschreibt den Vektor vom Urspung des Koordinatensystems $O(0|0|0)$ zu einem beliebigen Punkt $P(x|y|z)$.
\[	\boxed{ \vec{r} = \overrightarrow{OP} = x\vec{e_x} + y\vec{e_y} + z\vec{e_z} :=
	\left( 
	  \begin{array}{ccc} 
	    x \\ y \\ z
	  \end{array}
	\right) }
\]
\noindent
Die Vektoren $\vec{e_x},\vec{e_y},\vec{e_z}$ sind die Einheitsvektoren des Koordinatensystems (meist einfach 1 ohne Einheit).
\subsection{Länge eines Ortsvektors (Norm bzw. Betrag)}
\[ \boxed{ ||\vec{r}|| = r = \sqrt{x^2 + y^2 + z^2} } \]

\subsection{Länge eines Vektors (Norm bzw. Betrag)}
\[ \boxed{ ||\vec{a}|| = a = \overrightarrow{P_1P_2} = \sqrt{a_x^2 + a_y^2 + a_z^2} } \]
In dieser Form entspricht $a$ der Raumdiagonale im Quader zu $(a_x|a_y|a_z)$.

\subsection{Vektor aus Anfangs- und Endpunkt}
Möchte man den Vektor $\vec{a}$ von $P_1$ (Anfangspunkt) zu $P_2$ (Endpunkt) haben, so rechnet man Anfang - Ende, bzw. $\vec{P_2} - \vec{P_1}$.
\[  \boxed{
    \vec{a} = \overrightarrow{P_1P_2} = \vec{r_2} - \vec{r_1} =
    \left( 
	  \begin{array}{ccc} 
	    x_2 - x_1 \\ y_2 - y_1 \\ z_2 - z_1
	  \end{array}
	\right)
    }
\]

\subsection{Distanz zweier Punkte}
Um die Distanz von $P_1$ zu $P_2$ zu ermitteln, berechnet man die Norm von $\overrightarrow{P_1P_2}$.
\[ \boxed{
   \overline{P_1P_2} = \sqrt{ (x_2 - x_1)^2 + (y_2 - y_1)^2 + (z_2 + z_1)^2 }
   }
\]

\subsection{Mittelpunkt einer Strecke}
\[ \boxed{ \vec{r_M} = \frac{1}{2} \cdot (\vec{r_1} + \vec{r_2}) } \]
\[ \boxed{ \Rightarrow x_M = \frac{x_1 + x_2}{2} \\ y_m = \frac{y_1 + y_2}{2} \\ z_M = \frac{z_1 + z_2}{2} } \]

\section{Rechenoperationen mit Vektoren}

\subsection{Addition/Subtraktion}
\[ \boxed{ \vec{a}\pm\vec{b} =  
    \left( 
	  \begin{array}{ccc} 
	    a_x \\ a_y \\ a_z
	  \end{array}
	\right)
	\pm
	\left( 
	  \begin{array}{ccc} 
	    b_x \\ b_y \\ b_z
	  \end{array}
	\right)
	=
	\left( 
	  \begin{array}{ccc} 
	    a_x \pm b_x \\ a_y \pm b_y \\ a_z \pm b_z
	  \end{array}
	\right)
} \]

\subsection{Multiplikation mit Skalar}
\[ \boxed{ k \cdot \vec{a} = k \cdot 
\left( 
	  \begin{array}{ccc} 
	    a_x \\ a_y \\ a_z
	  \end{array}
	\right)
	=
	\left( 
	  \begin{array}{ccc} 
	    k \cdot a_x \\ k \cdot a_y \\ k \cdot a_z
	  \end{array}
	\right)
	} \\ \text{für } k \in \mathbb{R}
\]

\subsection{Skalarprodukt}
\[ \boxed{ \vec{a} \cdot \vec{b} = a \cdot b \cdot cos(\varphi) = a_x b_x + a_y b_y + a_z b_z } \]
Der Winkel $\varphi$ ist der Zwischenwinkel von $\vec{a}$ und $\vec{b}$.

\noindent
Für $\vec{a} \neq \vec{0}$, $\vec{b} \neq \vec{0}$ gilt: $\vec{a} \bot \vec{b} \Leftrightarrow \vec{a} \cdot \vec{b} = 0$!

\subsection{Winkel zwischen zwei Vektoren}
\[ \boxed{ cos \varphi = \frac{\vec{a} \cdot \vec{b} }{||\vec{a}|| \cdot ||\vec{b}||} } \]
\[ \boxed{ cos \varphi = \frac{a_x b_x + a_y b_y + a_z b_z}{ \sqrt{a_x^2 + a_y^2 + a_z^2} \sqrt{b_x^2 + b_y^2 + b_z^2} } } \]

\subsection{Vektorprodukt (Kreuzprodukt)}
Mit dem Vektorprodukt erhält man einen Vektor der senkrecht zur Ebene steht, also den Normalenvektor zur Ebene.

\[ \boxed{ \vec{c} = \vec{a} \times \vec{b} = 
\left( 
	  \begin{array}{ccc} 
	    a_x \\ a_y \\ a_z
	  \end{array}
	\right)
	\times
	\left( 
	  \begin{array}{ccc} 
	    b_x \\ b_y \\ b_z
	  \end{array}
	\right)
	=
	\left( 
	  \begin{array}{ccc} 
	    a_y b_z - a_z b_y \\ a_z b_x - a_x b_z \\ a_x b_y - a_y b_x
	  \end{array}
	\right)
} \]
\noindent
Die Fläche die von $\vec{a}$ und $\vec{b}$ aufgespannt wird, entspricht der Norm des Vektorprodukts $c=|\vec{a}\times\vec{b}| = a \cdot b \cdot sin \varphi$ .
Zu Beachten ist, dass $\vec{b} \times \vec{a} = -(\vec{a} \times \vec{b}) $

\subsection{Spatprodukt}
Das Spatprodukt entspricht dem Volumen welches von drei Vektoren aufgespannt wird.

\[ \boxed{ (\vec{a},\vec{b},\vec{c}) = (\vec{a} \times \vec{b}) \cdot \vec{c} = (\vec{b} \times \vec{c}) \cdot \vec{a} = (\vec{c} \times \vec{a}) \cdot \vec{b} } \]
