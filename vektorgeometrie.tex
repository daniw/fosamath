\section{Vektorgeometrie in der Ebene}

\subsection{Anstand zweier Puntke}
\[ \boxed{ \overline{P_1 P_2} = \sqrt{ (x_2 - x_1)^2 + (y_2 - y_1)^2 } } \]

\subsection{Geradengleichungen}

\subsubsection{Normalform (explizite Form)}
\[ \boxed{ g: y= mx + q }\]
\[ \boxed{ \text{Steigung } m = \frac{y_2 - y_1}{x_2 - x_1} = \frac{\Delta y}{\Delta x}  = tan \varphi } \]

\subsubsection{Koordinatenform (implizite Form)}
\[ \boxed{ g: ax + by + c = 0 } \]

\subsubsection{Achsenabschnittsform}
\[ \boxed{ g: \frac{x}{p} + \frac{y}{q} = 1 } \]

\subsubsection{Hesse'sche Normalform}
\[ \boxed{ g:  \frac{ax + by + c}{\sqrt{ a^2 + b^2 } } = 0 }  \]

\subsubsection{Parameterform}
\[ \boxed{ 
    g: \vec{r} = \vec{r_0} + t \cdot \vec{a}  = 
      \left( 
	\begin{array}{cc} 
	  x_0 \\ y_0
	\end{array}
      \right)
      + t \cdot 
      \left( 
	\begin{array}{cc} 
	  a_x \\ a_y
	\end{array}
      \right)  
   }
\]


\subsection{Normalenvektor}
Der Normalenvektor ist ein Vektor, welcher senkrecht auf einem anderen Vektor bzw. einer Geraden liegt. Hier im Beispiel in welchem $ \vec{n} \bot g(x)$
\[ \boxed{ \vec{n} = 
      \left( 
	\begin{array}{cc} 
	  n_x \\ n_y
	\end{array}
      \right)
      =
      \left( 
	\begin{array}{cc} 
	  a \\ b
	\end{array}
      \right)
      =
      \left( 
	\begin{array}{cc} 
	  -a_y \\ a_x
	\end{array}
      \right)
} \]
\noindent
Der Richtungsvektor von $g(x)$ ist 
$  
      \left( 
	\begin{array}{cc} 
	  a_x \\ a_y
	\end{array}
      \right)
      \Rightarrow 
      \left( 
	\begin{array}{cc} 
	  -a_y \\ a_x
	\end{array}
      \right)
      = \vec{n}
$.

\subsection{Abstand Punkt zu Gerade}
Für eine Gerade $g: ax + by + c = 0$ und einen Punkt $P_1 (x_1 | y_1)$ gilt:
\[ \boxed{ d = \left| \frac{ax_1 + by_1 + c}{\sqrt{a^2 + b^2}} \right| } \]

\subsection{Schnittwinkel zwischen Geraden}
Für den spitzen Schnittwinkel $\varphi$ zwischen den Geraden 

$g_1: y = m_1x + q_1$ und $g_2: y = m_2x + q_2$ gilt:
\[ \boxed{ tan\varphi = \left| \frac{m_2 - m_1}{1 + m_1 \cdot m_2} \right| } \\ \text{für } \varphi \neq 90^{\circ} \]

\[ \boxed{ g_1 || g_2 \Leftrightarrow m_1 = m_2 \text{und } g_1 \bot g_2 \Leftrightarrow m_2 = - \frac{1}{m_1} } \\ \text{für } m_1 \neq 0 \]

\section{Vektorgeometrie im Raum}

\subsection{Ortsvektor}
Ein Ortsvektor beschreibt den Vektor vom Urspung des Koordinatensystems $O(0|0|0)$ zu einem beliebigen Punkt $P(x|y|z)$.
\[	\boxed{ \vec{r} = \overrightarrow{OP} = x\vec{e_x} + y\vec{e_y} + z\vec{e_z} :=
	\left( 
	  \begin{array}{ccc} 
	    x \\ y \\ z
	  \end{array}
	\right) }
\]
\noindent
Die Vektoren $\vec{e_x},\vec{e_y},\vec{e_z}$ sind die Einheitsvektoren des Koordinatensystems (meist einfach 1 ohne Einheit).
\subsection{Länge eines Ortsvektors (Betrag)}
\[ \boxed{ |\vec{r}| = r = \sqrt{x^2 + y^2 1 z^2} } \]
