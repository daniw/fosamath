% coding:utf-8

%----------------------------------------
%FOSAMATH, a LaTeX-Code for a mathematical summary for basic analysis
%Copyright (C) 2013, Daniel Winz, Ervin Mazlagic, Adrian Imboden, Philipp Langer

%This program is free software; you can redistribute it and/or
%modify it under the terms of the GNU General Public License
%as published by the Free Software Foundation; either version 2
%of the License, or (at your option) any later version.

%This program is distributed in the hope that it will be useful,
%but WITHOUT ANY WARRANTY; without even the implied warranty of
%MERCHANTABILITY or FITNESS FOR A PARTICULAR PURPOSE.  See the
%GNU General Public License for more details.
%----------------------------------------


% coding:utf-8
\section{Folgen}
$a_1$ (bzw. $a_0$) ist das erste Glied einer Folge\\
$a_n$ ist das $n$-te Glied einer Folge

\subsection{Form}
Für die Form einer Folge existieren drei Spezialfälle. Für 
$(a_n)_{n \in \mathbb{N}}$ gilt die Folge je nach dem als
\subsubsection*{monoton wachsend}
$ \boxed{ a_{n+1} \geq a_n \quad \forall \quad n \in \mathbb{N} } $ Bsp.: $a_n 
= 2^n$, $a_{n+1} = 2^{n+1}$
\subsubsection*{monoton fallend}
$ \boxed{ a_{n+1} \leq a_n \quad \forall \quad n \in \mathbb{N} } $ Bsp.: $a_n 
= \frac{1}{n}$, $a_{n+1} = \frac{1}{n+1}$
\subsubsection*{alternierend}
$ \boxed{ a_{n+1} \cdot a_n < 0 } $ Bsp.: $a_n := (-2)^n$, $a_{n+1} 
= (-2)^{n+1} \cdot (-2)^n = (-2)^{2n+1}$\\\\
Bei den monoton wachsend/fallenden Folgen kann noch weiter unterschieden werden 
zwischen streng monoton und einfach monoton. Streng monoton bedeutet dann, dass 
$a_n \neq a_{n+1}$ wobei das bei einfach monotonen der Fall sein kann 
(vgl.~\ref{subsec:monotonie}).
\subsection{rekursive Darstellung}
Bei der rekursiven Darstellung wird eine Folge durch einen Startwert und eine 
Abbildungsvorschrift dargestellt. \\
\[ \boxed{ \begin{matrix}
\text{Startwert} & f(1) = a_1 \\
\text{Vorschrift} & F(a_n, \ldots, a_{n-k}) := f(n + 1) = a_{n + 1} 
\quad \forall ~ k=konst
\end{matrix}} \]
Ist $k > 1$, so müssen mehrere Startwerte festgelegt werden. 

\subsection{arithmetische Folgen}
$d$ ist die Differenz zwischen zwei benachbarten Gliedern\\
\[ \boxed{d = a_{n+1} - a_n} \]
\[ \boxed{a_{n+1} = a_n + d} \]
\[ \boxed{ \begin{matrix} 
a_0 :& a_n =& a_0 + n \cdot d \\
a_1 :& a_n =& a_1 + (n - 1)d 
\end{matrix}}\]

\subsection{geometrische Folgen}
$q$ ist der Quotient von zwei benachbarten Gliedern\\
\[ \boxed{\frac{a_{n+1}}{a_n} = q} \]
\[ \boxed{a_{n+1} = a_n \cdot q} \]
\[ \boxed{\begin{array}{ll}
a_0 :& a_n = q^{n} a_0\\
a_1 :& a_n = q^{n-1} a_1
\end{array}} \]

\ifti
\subsubsection{Folgen mit dem TI-89}
%seq($a_n$, Variable, $n_1$, $n_2$)\\\\
\verb{seq(EXP,VAR,LOW,HIGH[,STEP]){\\\\
\begin{tabular}{@{}lll}
EXP	& Ausdruck	& bezeichnet den Term \\
VAR 	& Variable	& bezeichnet die inkrementierte Variabel \\
LOW	& untere Grenze	& Anfangspunkt der Inkrementierung \\
HIGH	& obere Grenze	& Endpunkt der Inkrementierung \\
STEP	& n-Schritte	& Zeigt Resultate mit n-ten Schritt
\end{tabular}\\\\
Bsp.: \verb{seq(1/x,x,1,10,5){ erzeugt die Ausgabe \verb?{1   1/6}?
\fi
\ifnspire
\subsubsection{Folgen mit dem TI-Nspire}
seq($a_n$, Variable, $n_1$, $n_2$)\\\\
\begin{tabular}{@{}ll}
$a_n$    & Formel für die Berechnung des n-ten Gliedes\\
Variable & Variable in der Formel\\
$n_1$    & erstes zu berechnendes Glied\\
$n_2$    & letztes zu berechnendes Glied
\end{tabular}
\fi

\subsection{Konvergenz von Folgen}
Eine Folge ist konvergent, wenn sich die Folge mit steigendem Index einer Zahl 
annähert. Diese Zahl wird Grenzwert genannt. 
\[ \boxed{\lim\limits_{n \to \infty} a_n = a \quad a \in \mathbb{R}}\]

\subsubsection{Rechenregeln}
$ a_n \xrightarrow[\rightarrow \infty]{} a $, 
$ b_n \xrightarrow[n \rightarrow \infty]{} b $
\[ \boxed{ a_n + b_n \xrightarrow[]{n \rightarrow \infty} a + b } \]
\[ \boxed{ a_n - b_n \xrightarrow[]{n \rightarrow \infty} a - b } \]
\[ \boxed{ a_n \cdot b_n \xrightarrow[]{n \rightarrow \infty} a \cdot b } \]
\[ \boxed{ \alpha \cdot a_n \xrightarrow[]{n \rightarrow \infty} \alpha \cdot a, 
\quad \alpha \in \mathbb{R} } \]
\[ \boxed{ \frac{a_n}{b_n} \xrightarrow[]{n \rightarrow \infty} \frac{a}{b}, 
\quad \text{falls} \quad b_n \neq 0 \quad \forall \quad n \in \mathbb{N} } \]

\ifti
\subsubsection{Konvergenz von Folgen mit dem TI-89}
Für die Berechnung von Grenzwerten mit dem TI-89 siehe \ref{subsubsec:limti}. 
\fi
\ifnspire
\subsubsection{Konvergenz von Folgen mit dem TI-Nspire}
Für die Berechnung von Grenzwerten mit dem TI-Nspire siehe 
\ref{subsubsec:limnspire}. 
\fi
