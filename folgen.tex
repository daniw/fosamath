% coding:utf-8
\section{Folgen}

\subsection{rekursive Darstellung}
Bei der rekursiven Darstellung wird eine Folge durch einen Startwert und eine Abbildungsvorschrift dargestellt. \\
\[ \boxed{ \begin{matrix}
\text{Startwert} & f(1) = a_1 \\
\text{Vorschrift} & F(f(1), f(2), \ldots, f(n) := f(n + 1) = a_{n + 1})
\end{matrix}} \]

\subsection{arithmetische Folgen}
\[ \boxed{a_{n+1} - a_n = d} \]
\[ \boxed{a_{n+1} = a_n + d} \]
\[ \boxed{a_n = a_1 + (n - 1)d} \]

\subsection{geometrische Folgen}
\[ \boxed{\frac{a_{n+1}}{a_n} = q} \]
\[ \boxed{a_{n+1} = a_n * q} \]
\[ \boxed{a_n = q^{n-1} a_1} \]

\subsection{Konvergenz von Folgen}
