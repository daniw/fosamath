% coding:utf-8
\section{Folgen}
$a_1$ ist das erste Glied einer Folge\\
$a_n$ ist das $n$-te Glied einer Folge

\subsection{Form}
Eine Folge kann drei spezielle Formen annehmen. Für $(a_n)_{n \in \mathbb{N}}$ gilt die Folge je
nach den als
\subsubsection*{monoton wachsend}
$ \boxed{ a_{n+1} \geq a_n \quad \forall \quad n \in \mathbb{N} } $ Bsp.: $a_n = 2^n$, $a_{n+1} = 2^{n+1}$
\subsubsection*{monoton fallend}
$ \boxed{ a_{n+1} \leq a_n \quad \forall \quad n \in \mathbb{N} } $ Bsp.: $a_n = \frac{1}{n}$, $a_{n+1} = \frac{1}{n+1}$
\subsubsection*{alternierend}
$ \boxed{ a_{n+1} \cdot a_n < 0 } $ Bsp.: $a_n := (-2)^n$, $a_{n+1} = (-2)^{n+1} \cdot (-2)^n = (-2)^{2n+1}$

\subsection{rekursive Darstellung}
Bei der rekursiven Darstellung wird eine Folge durch einen Startwert und eine Abbildungsvorschrift dargestellt. \\
\[ \boxed{ \begin{matrix}
\text{Startwert} & f(1) = a_1 \\
\text{Vorschrift} & F(f(1), f(2), \ldots, f(n)) := f(n + 1) = a_{n + 1}
\end{matrix}} \]

\subsection{arithmetische Folgen}
$d$ ist die Differenz zwischen zwei benachbarten Gliedern\\
\[ \boxed{d = a_{n+1} - a_n} \]
\[ \boxed{a_{n+1} = a_n + d} \]
\[ \boxed{ \begin{matrix} 
a_0 :& a_n =& a_0 + n \cdot d \\
a_1 :& a_n =& a_1 + (n - 1)d 
\end{matrix}}\]

\subsection{geometrische Folgen}
$q$ ist der Quotient von zwei benachbarten Gliedern\\
\[ \boxed{\frac{a_{n+1}}{a_n} = q} \]
\[ \boxed{a_{n+1} = a_n \cdot q} \]
\[ \boxed{\begin{array}{ll}
a_0 :& a_n = q^{n} a_0\\
a_1 :& a_n = q^{n-1} a_1
\end{array}} \]

\ifti
\subsubsection{Folgen mit dem TI-89}
%seq($a_n$, Variable, $n_1$, $n_2$)\\\\
\verb{seq(EXP,VAR,LOW,HIGH[,STEP]){\\\\
\begin{tabular}{@{}lll}
EXP	& Ausdruck	& bezeichnet den Term \\
VAR 	& Variable	& bezeichnet die inkrementierte Variabel \\
LOW	& untere Grenze	& Anfangspunkt der Inkrementierung \\
HIGH	& obere Grenze	& Endpunkt der Inkrementierung \\
STEP	& n-Schritte	& Zeigt Resultate mit n-ten Schritt
\end{tabular}\\\\
Bsp.: \verb{seq(1/x,x,1,10,5){ erzeugt die Ausgabe \verb?{1   1/6}?
\fi
\ifnspire
\subsubsection{Folgen mit dem TI-Nspire}
seq($a_n$, Variable, $n_1$, $n_2$)\\\\
\begin{tabular}{@{}ll}
$a_n$    & Formel für die Berechnung des n-ten Gliedes\\
Variable & Variable in der Formel\\
$n_1$    & erstes zu berechnendes Glied\\
$n_2$    & letztes zu berechnendes Glied
\end{tabular}
\fi

\subsection{Konvergenz von Folgen}
Eine Folge ist konvergent, wenn der Grenzwert der Folge eine reelle Zahl ist
\[ \boxed{\lim\limits_{n \to \infty} a_n = a \quad a \in \mathbb{R}}\]

\subsubsection{Rechenregeln}
$ a_n \xrightarrow[\rightarrow \infty]{} a $, $ b_n \xrightarrow[n \rightarrow \infty]{} b $
\[ \boxed{ a_n + b_n \xrightarrow[]{n \rightarrow \infty} a + b } \]
\[ \boxed{ a_n - b_n \xrightarrow[]{n \rightarrow \infty} a - b } \]
\[ \boxed{ a_n \cdot b_n \xrightarrow[]{n \rightarrow \infty} a \cdot b } \]
\[ \boxed{ \alpha \cdot a_n \xrightarrow[]{n \rightarrow \infty} \alpha \cdot a, \quad \alpha \in \mathbb{R} } \]
\[ \boxed{ \frac{a_n}{b_n} \xrightarrow[]{n \rightarrow \infty} \frac{a}{b}, \quad \text{falls} \quad b_n \neq 0 \quad \forall \quad n \in \mathbb{N} } \]

\ifti
\subsubsection{Konvergenz von Folgen mit dem TI-89}
Für die Berechnung von Grenzwerten mit dem TI-89 siehe \ref{subsubsec:limti}. 
\fi
\ifnspire
\subsubsection{Konvergenz von Folgen mit dem TI-Nspire}
Für die Berechnung von Grenzwerten mit dem TI-Nspire siehe \ref{subsubsec:limnspire}. 
\fi