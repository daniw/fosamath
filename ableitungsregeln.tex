% coding:utf-8
\section{Ableitungsregeln}

\subsection{Schreibweisen}
\[ \boxed{f(x)' = \frac{d}{dx} f(x)} \]

\subsection{Stetigkeit}
Eine Funktion wird als stetig bezeichnet, wenn die Funktion keine 
Sprungstellen\footnote{Salopp gesagt heisst dies, dass die Funktion ohne Absetzten des Stiftes gezeichnet werden kann.}
hat (für das MATH-Modul ist diese Definition ausreichend).

\subsection{Differenzierbarkeit}
Der Begriff Differenzierbarkeit sagt aus, dass wenn eine Funktion
differenzierbar ist, zu jedem Punkt genau eine Tangente existiert. 
Damit dies möglich ist, muss die Funktion stetig sein.\\\\
Bsp.: $f(x)=|x|$ ist stetig aber nicht differenzierbar in $x=0$, 
denn vom positiven Bereich aus betrachtet hat die Funktion die Steigung 1.
Vom negativen Bereich aus betrachtet hat die Funktion die Steigung -1. 
Im Punkt $x=0$ ist somit nicht klar, welche Steigung gilt.\\\\
Bsp.: 
$ \quad \quad f(x) = \left\lbrace 
\begin{array}{ll}
	\dfrac{\alpha \cdot x}{\xi} & x \leq \gamma \\
	\dfrac{\beta \cdot x}{\psi}  & x > \gamma 
\end{array} \right.  $ \\
In diesem Beispiel ist die Funktion $f$ definiert hin zu $\gamma$ 
(von rechts und von links aus).
Soll die Funktion an der Stelle $\gamma$ differenzierbar sein, 
so müssen in $\gamma$ die Funktionswerte und Steigungen von beiden Seiten her
identisch sein. \\
D.h. im Punkt $x=\gamma$ gilt, dass 
$\frac{\alpha \cdot x}{\xi} = \frac{\beta \cdot x}{\psi}$ und
$\left(\frac{\alpha \cdot x}{\xi}\right)' = 
 \left(\frac{\beta \cdot x}{\psi} \right)'$ ist.

\subsection{Grundoperationen}

\subsubsection{Summenregel}
% \[ \boxed{ (f(x) + g(x))' = f'(x) + g'(x) } \]
\[ \boxed{ (u(x) + v(x))' = u'(x) + v'(x) } \]
\[ \text{Wichtig: Ableitung einer konstanten Funktion ist Null! } \]
% \[ \Rightarrow (f(x) + c)' = f'(x) \text{ für } c \in R \]
\[ \Rightarrow (u(x) + c)' = u'(x) \text{ für } c \in R \]

\subsubsection{Faktorregel}
% \[ \boxed{ (c \cdot f(x))' = c \cdot f'(x) } \]
\[ \boxed{ (c \cdot u(x))' = c \cdot u'(x) } \]
\[ \text{Ein konstanter Faktor bleibt beim Differenzieren (Ableiten) erhalten!} \]

\subsubsection{Produkteregel}
% \[ \boxed{ (f(x) \cdot g(x))' = f'(x) \cdot g(x) + f(x) \cdot g'(x) } \]
\[ \boxed{ (u(x) \cdot v(x))' = u'(x) \cdot v(x) + u(x) \cdot v'(x) } \]

\subsubsection{Quotientenregel}
% \[ \boxed{ \left( \frac{f(x)}{g(x)} \right)' = \frac{ f'(x) \cdot g(x) - f(x) \cdot g'(x) }{ g^2(x) } } \\ \text{ gilt falls }g(x) \neq 0 \text{ !} \]
\[ \boxed{ \left( \frac{u(x)}{v(x)} \right)' = \frac{ u'(x) \cdot v(x) - u(x) \cdot v'(x) }{ v^2(x) } } \\ \text{ gilt falls }v(x) \neq 0 \text{ !} \]

\subsubsection{Kettenregel}
% \[ \boxed{ (f(g(x)))' = g'(x) \cdot f'(g(x)) } \]
\[ \boxed{ (u(v(x)))' = v'(x) \cdot u'(v(x)) } \]

\subsection{Ableitung von Kurven in Parameterdarstellung}\label{subsec:ableitungsregeln}
\[ \boxed{\gamma(t) = \left(\begin{matrix}x(t)\\y(t)\end{matrix}\right) \qquad \Rightarrow \qquad \dot{\gamma}(t) = \left(\begin{matrix}\dot{x}(t)\\\dot{y}(t)\end{matrix}\right)} \]
Falls $\gamma$ keine vertikalen Tangenten besitzt, so kann man die Spur von $\gamma$ (d.h. das Schaubild) lokal als Graph einer differenzierbar reellwertigen Funktion auffassen. Wir können also schreiben 
\[ y(x(t)) = y(t) \]
und damit gilt: 
\[ \frac{d}{dt}y(x(t)) = \frac{d}{dt}y(t), ~\Leftrightarrow~ y'(x(t)) \dot{x}(t) = \dot{y}(t), ~\Leftrightarrow~ y'(x(t)) = \frac{\dot{y}(t)}{\dot{x}(t)} \]

% \[ \boxed{\frac{d}{dt}\left(\begin{matrix}x(t)\\y(t)\end{matrix}\right) = \frac{\dot{y}(t)}{\dot{x}(t)}} \]


\subsection{Implizites Ableiten}
Eine implizite Funktion F ist definiert durch das Nullstellengebilde 
\[ F(x,y) = 0 \]
Manchmal kann man eine Funktion $f(x) = (y(x))$ finden. 
\[ F(x, f(x)) = 0 \]
Soll nun eine Tangente an diese Funktion gelegt werden, kann auch hier die Tangentengleichnung (\ref{subsec:tangentengleichung}) angewendet werden. 
\[ T(x) = f'(x_0)(x - x_0) + f(x_0) \]
$x_0$ und $f(x_0)$ sind durch den Punkt, an welchem die Tangente angelegt werden soll vorgegeben. Um $f'(x_0)$ zu erhalten muss die Funktion nach x abgeleitet werden. Dafür ist y als $f(x)$ zu betrachten und die Kettenregel anzuwenden. Anschliessend wird die erhaltene Gleichung nach $f'(x)$ aufgelöst und kann dann in die Tangentengleichnung eingesetzt werden. 

% Implizite Gleichungen bzw. Funktionen sind Funktionen welche die Form $x^a + d x y + y^b + c= 0$ haben.
% Gegenüber der expliziten Form $f(x)=y=...$ ist $y$ selbst als Funktion von $x$ definiert.\\\\
% Bsp.: Aus einer implizit formulierten Gleichung eine Tangente an einem bestimmten Punkt legen.
% \[ x^a + d x y + y^b + c = 0, \quad T_{(x_0)} = f'(x_0)(x-x_0)+f(x_0), \quad P(x_p|y_p) \]
% Zuerst muss der Ausdruck $y$ durch $y(x)$ ersetzt werden.
% \[ x^a + d x y + y(x)^b = 0\]
% Danach kann abgeleitet und nach $y'(x)$ aufgelöst werden.
% \[ \begin{array}{lll@{\quad}}
% ax^{a-1} + d\Big(y(x) + xy'(x)\Big) + y'(x)by(x)^{b-1} =0 & |& \text{ausklammern} \\ 
% ax^{a-1} + dy(x) + axy'(x) + y'(x)by(x)^{b-1} =0          & |& -ax^{a-1} - dy(x) \\
% dxy'(x) + y'(x) by(x)^{b-1} = -ax^{a-1} - dy(x)           & |& y'(x) \text{ ausklammern } \\
% y'(x) \cdot (dx + by^{b-1}) = -ax^{a-1} - dy(x)           & |& \div (dx+by^{b-1}) \\
% y'(x) = \dfrac{-ax^{a-1} - dy(x)}{dx + by^{b-1}}          &  & \\
% \end{array} \]
% Nun kann man die Tangentengleichung bilden:
% \[ T(x) = y'(x_p) (x-x_p) + y(x_p) \]

\newpage

\subsection{Spezielle Regeln}

\subsubsection{Exponenten}
\[ \boxed{ (x^n)' = n\cdot x^{(n-1)} } \]
\[ \boxed{ (e^x)' = e^x } \]
\[ \boxed{ (e^{k\cdot x})' = k \cdot e^{k\cdot x} } \]
\[ \boxed{ (a^x)' = \ln_a (a^x) } \]

\subsubsection{Logarithmen}
\[ \boxed{ (\ln(x))' = \frac{1}{x} } \]  
% \[ \boxed{ (a_{log_x})' = \frac{1}{x \cdot ln(a)} } \]

\subsubsection{Trigonometrie}
% \[ \boxed{ (\sin(x))' = \cos(x) } \]  
% \[ \boxed{ (\cos(x))' = -\sin(x) } \] 
% \[ \boxed{ (\tan(x))' = \frac{1}{\cos^2(x)} } \]  
% \[ \boxed{ (\cot(x))' = -\frac{1}{\sin^2(x)} } \]

\[ \boxed{ \begin{array}{ l @{}l @{\quad\quad\quad}l @{}l }
(\sin(x))' =   &\cos(x)             & \quad (\arcsin(x))' =   & \dfrac{1}{\sqrt{1 - x^2}} \\\\
(\cos(x))' = - &\sin(x)             & \quad (\arccos(x))' = - & \dfrac{1}{\sqrt{1 - x^2}} \\\\
(\tan(x))' =   &\dfrac{1}{\cos^2(x)} & \quad (\arctan(x))' =   & \dfrac{1}{1 + x^2} \\\\
(\cot(x))' = - &\dfrac{1}{\sin^2(x)} & \quad (\arccot(x))' = - & \dfrac{1}{1 + x^2} 
\end{array}} \]

\subsubsection{Hyperbolicus}
\[ \boxed{(\sinh(x))' = \cosh(x)} \]
\[ \boxed{(\cosh(x))' = \sinh(x)} \]
