% coding:utf-8
\section{Ableitungsregeln}

\subsection{Schreibweisen}
\[ \boxed{f(x)' = \frac{d}{dx} f(x) = \dot{x}} \]

\subsection{Stetigkeit}
Eine Funktion wird als stetig bezeichnet, wenn die Funktion keine 
Sprungstellen\footnote{Salopp gesagt heisst dies, dass die Funktion ohne Absetzten des Stiftes gezeichnet werden kann.}
hat (für das MATH-Modul ist diese Definition ausreichend).

\subsection{Differenzierbarkeit}
Der Begriff Differenzierbarkeit sagt aus, dass wenn eine Funktion
differenzierbar ist, zu jedem Punkt genau eine Tangente existiert. 
Damit dies möglich ist, muss die Funktion stetig sein.\\\\
Bsp.: $f(x)=|x|$ ist stetig aber nicht differenzierbar in $x=0$, 
denn vom positiven Bereich aus betrachtet hat die Funktion die Steigung 1.
Vom negativen Bereich aus betrachtet hat die Funktion die Steigung -1. 
Im Punkt $x=0$ ist somit nicht klar, welche Steigung gilt.\\\\
Bsp.: 
$ \quad \quad f(x) = \left\lbrace 
\begin{array}{ll}
	\dfrac{\alpha \cdot x}{\xi} & x \leq \gamma \\
	\dfrac{\beta \cdot x}{\psi}  & x > \gamma 
\end{array} \right.  $ \\
In diesem Beispiel ist die Funktion $f$ definiert hin zu $\gamma$ 
(von rechts und von links aus).
Soll die Funktion an der Stelle $\gamma$ differenzierbar sein, 
so müssen in $\gamma$ die Funktionswerte und Steigungen von beiden Seiten her
identisch sein. \\
D.h. im Punkt $x=\gamma$ gilt, dass 
$\frac{\alpha \cdot x}{\xi} = \frac{\beta \cdot x}{\psi}$ und
$\left(\frac{\alpha \cdot x}{\xi}\right)' = 
 \left(\frac{\beta \cdot x}{\psi} \right)'$ ist.

\subsection{Grundoperationen}

\subsubsection{Summenregel}
% \[ \boxed{ (f(x) + g(x))' = f'(x) + g'(x) } \]
\[ \boxed{ (u(x) + v(x))' = u'(x) + v'(x) } \]
\[ \text{Wichtig: Ableitung einer konstanten Funktion ist Null! } \]
% \[ \Rightarrow (f(x) + c)' = f'(x) \text{ für } c \in R \]
\[ \Rightarrow (u(x) + c)' = u'(x) \text{ für } c \in R \]

\subsubsection{Faktorregel}
% \[ \boxed{ (c \cdot f(x))' = c \cdot f'(x) } \]
\[ \boxed{ (c \cdot u(x))' = c \cdot u'(x) } \]
\[ \text{Ein konstanter Faktor bleibt beim Differenzieren (Ableiten) erhalten!} \]

\subsubsection{Produkteregel}
% \[ \boxed{ (f(x) \cdot g(x))' = f'(x) \cdot g(x) + f(x) \cdot g'(x) } \]
\[ \boxed{ (u(x) \cdot v(x))' = u'(x) \cdot v(x) + u(x) \cdot v'(x) } \]

\subsubsection{Quotientenregel}
% \[ \boxed{ \left( \frac{f(x)}{g(x)} \right)' = \frac{ f'(x) \cdot g(x) - f(x) \cdot g'(x) }{ g^2(x) } } \\ \text{ gilt falls }g(x) \neq 0 \text{ !} \]
\[ \boxed{ \left( \frac{u(x)}{v(x)} \right)' = \frac{ u'(x) \cdot v(x) - u(x) \cdot v'(x) }{ v^2(x) } } \\ \text{ gilt falls }v(x) \neq 0 \text{ !} \]

\subsubsection{Kettenregel}
% \[ \boxed{ (f(g(x)))' = g'(x) \cdot f'(g(x)) } \]
\[ \boxed{ (u(v(x)))' = v'(x) \cdot u'(v(x)) } \]

\newpage

\subsection{Spezielle Regeln}

\subsubsection{Exponenten}
\[ \boxed{ (x^n)' = n\cdot x^{(n-1)} } \]
\[ \boxed{ (e^x)' = e^x } \]
\[ \boxed{ (e^{k\cdot x})' = k \cdot e^{k\cdot x} } \]
\[ \boxed{ (a^x)' = ln_a (a^x) } \]

\subsubsection{Logarithmen}
\[ \boxed{ (ln(x))' = \frac{1}{x} } \]  
% \[ \boxed{ (a_{log_x})' = \frac{1}{x \cdot ln(a)} } \]

\subsubsection{Trigonometrie}
\[ \boxed{ (\sin(x))' = \cos(x) } \]  
\[ \boxed{ (\cos(x))' = -\sin(x) } \] 
\[ \boxed{ (\tan(x))' = \frac{1}{\cos^2(x)} } \]  
\[ \boxed{ (\cot(x))' = -\frac{1}{\sin^2(x)} } \]

\subsubsection{Hyperbolicus}
\[ \boxed{(sinh(x))' = cosh(x)} \]
\[ \boxed{(cosh(x))' = sinh(x)} \]

\subsection{Ableitung von Kurven}
\[ \boxed{\frac{d}{dx}\left(\begin{matrix}x(t)\\y(t)\end{matrix}\right) = \left(\begin{matrix}\dot{y}(t)\\\dot{x}(t)\end{matrix}\right)} \]