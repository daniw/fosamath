%\documentclass[a4paper,landscape]{article}
\documentclass[a4paper]{article}

\usepackage[a4paper]{geometry}

\pagestyle{empty}
\usepackage[pdftex]{graphicx}
%\usepackage{amsmath}
\usepackage{amssymb}
\usepackage[german]{babel}
\usepackage{amsfonts}
\usepackage[intlimits]{amsmath}
%\usepackage{theorem}

\include{macros}

\begin{document}

\noindent \textbf{HSLU MA+PHY1\_T \hspace{\stretch{1}} Peter Scheiblechner}

\begin{center}
\huge \textbf{Mathematik/Physik I} \\
\vspace{.1cm}
\LARGE \textbf{Tabelle von Ans\"atzen} \\
\Large \textbf{Lineare Differentialgleichungen zweiter Ordnung mit konstanten Koeffizienten} \\
\end{center}

Wir betrachten eine inhomogene lineare Differentialgleichung zweiter Ordnung mit konstanten Koeffizienten
$$
a_2y''+a_1y'+a_0y=s(x),
$$
wobei $a_0,a_1,a_2$ Konstanten und $s(x)$ die St\"orfunktion sind. Es geht hier um die Bestimmung einer partikul\"aren L\"osung $y_p(x)$
mit dem Ansatzverfahren. 
Wir geben eine Tabelle an, die f\"ur wichtige St\"orfunktionen den richtigen Ansatz angibt.
Der richtige Ansatz h\"angt auch von der zugeh\"origen homogenen Differentialgleichung
$$
a_2y''+a_1y'+a_0y=0
$$
ab. Die {\em charakteristische Gleichung} der homogenen Gleichung ist
$$
a_2k^2+a_1k+a_0=0.
$$
Die L\"osungen der charakteristischen Gleichung sind
$$
k_{1,2}=\frac{-a_1\pm\sqrt{a_1^2-4a_2a_0}}{2a_2}.
$$
Eine L\"osung $k=k_1$ der Gleichung heisst {\em einfache} L\"osung, falls $k_2\ne k$ ist. Die L\"osung $k=k_1$ heisst {\em zweifache} L\"osung, falls
$k=k_1=k_2$ ist. Eine Zahl $k$, die {\em keine} L\"osunge der Gleichung ist, heisst auch {\em nullfache} L\"osung.

\paragraph{Notationen}
Ein Polynom $n$-ten Grades ist eine Funktion der Form
$$
P_n(x)=\a_n x^n+\a_{n-1} x^{n-1}+\cdots+\a_1 x+\a_0,
$$
wobei $\a_0,\ldots,\a_n$ beliebige Konstanten sind. In der St\"orfunktion sind diese Konstanten bekannt, d.h.
vorgegeben. Im Ansatz sind es unbekannte Variablen, deren Werte man bestimmt, indem man den Ansatz mitsamt seinen
Ableitungen in die Differentialgleichung einsetzt und Koeffizienten vergleicht.

Wir verwenden f\"ur alle Konstanten in der St\"orfunktion griechische Buchstaben, also z.B. $\a_0,\ldots,\a_n,\la,\om$.
F\"ur die unbekannten Variablen im Ansatz verwenden wir lateinische Grossbuchstaben.

\paragraph{Bemerkungen}
\begin{enumerate}
	\item Eigentlich sind nur die letzten beiden Zeilen der Tabelle relevant. Sie umfassen alle anderen, z.B. ist f\"ur $\lambda=0$ und $n=0$ die
	St\"orfunktion in der letzten Zeile
	$$
	P_n(x)e^{\lambda x}\cos \omega x=A_0\cos \omega x.
	$$
	Oder f\"ur $\lambda=\om=0$ ist
	$$
	P_n(x)e^{\lambda x}\cos \omega x=P_n(x).
	$$
	\item Wenn in der St\"orfunktion mehrere Terme aus der Tabelle vorkommen, summiert man die Ans\"atze f\"ur jeden Term auf
	{\bf und streicht Terme mit gleichem Funktionstyp!} Wenn die St\"orfunktion z.B.~$s(x)=2x\sin 3x-5\cos 3x$ ist, so lautet der Ansatz
	$$
	y_p(x)=(A_1x+A_0)\sin 3x+(B_1x+B_0)\cos 3x.
	$$
	Dies ist der gleiche Ansatz, den man nur f\"ur den Term $2x\sin 3x$ bekommt. Der Ansatz f\"ur den Term $-5\cos 3x$ w\"are $A\sin 3x+B\cos 3x$,
	und diese Funktionstypen kommen im Ansatz bereits vor.
%	Oder wenn die St\"orfunktion $s(x)=x\sin 2x+x^2\cos 2x$ ist, so lautet der Ansatz
%	$$
%	y_p(x)=(A_2x^2+A_1x+A_0)\sin 2x+(B_2x^2+B_1x+B_0)\cos 2x.
%	$$
\end{enumerate}

\bigskip

\def\arraystretch{1.5}
\begin{tabular}{p{0.2\textwidth}|p{0.7\textwidth}}
%\begin{tabular}{ll}
\hline
St\"orfunktion $s(x)$ & Ansatz $y_p$ \\
\hline
$e^{\lambda x}$ & 
$Ax^me^{\lambda x}$, falls $\la$ $m$-fache L\"osung der charakteristischen Gleichung \\
$\sin \omega x$ & $Ax^m\sin \omega x+Bx^m\cos\omega x$, falls $j\om$ $m$-fache L\"osung der charakteristischen Gleichung \\
$\cos \omega x$ & $Ax^m\sin \omega x+Bx^m\cos\omega x$, falls $j\om$ $m$-fache L\"osung der charakteristischen Gleichung \\
$e^{\lambda x}\sin \omega x$ & $Ax^me^{\lambda x}\sin \omega x+Bx^me^{\lambda x}\cos\omega x$, falls $\la+j\om$ $m$-fache L\"osung der charakteristischen Gleichung \\
$e^{\lambda x}\cos \omega x$ & $Ax^me^{\lambda x}\sin \omega x+Bx^me^{\lambda x}\cos\omega x$, falls $\la+j\om$ $m$-fache L\"osung der charakteristischen Gleichung \\
$P_n(x)$ & $x^m(A_n x^{n}+A_{n-1} x^{n-1}+\cdots+A_1 x+A_0)$, falls $0$ $m$-fache L\"osung der charakteristischen Gleichung \\
$P_n(x)\sin \omega x$ & $x^m(A_n x^n+A_{n-1} x^{n-1}+\cdots+A_1 x+A_0)\sin \omega x$ \\
 & \qquad$+x^m(B_n x^n+B_{n-1} x^{n-1}+\cdots+B_1 x+B_0)\cos\omega x$, falls $j\om$ $m$-fache L\"osung der charakteristischen Gleichung \\
$P_n(x)\cos \omega x$ & $x^m(A_n x^n+A_{n-1} x^{n-1}+\cdots+A_1 x+A_0)\sin \omega x$ \\
 & \qquad$+x^m(B_n x^n+B_{n-1} x^{n-1}+\cdots+B_1 x+B_0)\cos\omega x$, falls $j\om$ $m$-fache L\"osung der charakteristischen Gleichung \\
$P_n(x)e^{\lambda x}\sin \omega x$ & $x^m(A_n x^n+A_{n-1} x^{n-1}+\cdots+A_1 x+A_0)e^{\lambda x}\sin \omega x$ \\
 & \qquad$+x^m(B_n x^n+B_{n-1} x^{n-1}+\cdots+B_1 x+B_0)e^{\lambda x}\cos\omega x$, falls $\la+j\om$ $m$-fache L\"osung der charakteristischen Gleichung \\
$P_n(x)e^{\lambda x}\cos \omega x$ & $x^m(A_n x^n+A_{n-1} x^{n-1}+\cdots+A_1 x+A_0)e^{\lambda x}\sin \omega x$ \\
 & \qquad$+x^m(B_n x^n+B_{n-1} x^{n-1}+\cdots+B_1 x+B_0)e^{\lambda x}\cos\omega x$, falls $\la+j\om$ $m$-fache L\"osung der charakteristischen Gleichung \\
\end{tabular}


\end{document}
