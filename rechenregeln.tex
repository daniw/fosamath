% coding:utf-8
\section{Rechenregeln}
\subsection{Bruchrechnen}
\[ \boxed{\frac{a}{b} \cdot \frac{c}{d} = \frac{a \cdot c}{b \cdot d}} \]
\[ \boxed{n \frac{a}{b} = \frac{n \cdot a}{b}} \]
\[ \boxed{\frac{a}{b} : \frac{c}{d} = \frac{a}{b} \cdot \frac{d}{c} = \frac{a \cdot d}{b \cdot c}} \]

\subsection{Polynomdivision}
Beispiel: 
% \[ \boxed{\begin{array}{rrrlrl}
% &(9x^3 &- 6&x^2 &- 8x&):(3x + 4) = \underline{\underline{3x^2 + 2x}}\\
% -&(9x^3 &- 12&x^2)&&\\
% &&6&x^2 &- 8x&\\
% &&-(6&x^2 &- 8x&)\\
% &&&&0&
% \end{array}} \]

\begin{tabular}{|r@{}r@{}r@{}r@{}l@{}r@{}r@{}l|}
\hline
\rule{0pt}{12pt}&&$(9x^3 $&$- 6$&$x^2 $&$- 8x$&$)$&$:(3x + 4) = \underline{\underline{3x^2 + 2x}}$\\
&$-$&$(9x^3 $&$- 12$&$x^2)$&$\downarrow\,\,$&&\\
\cline{2-5}\rule{0pt}{12pt}&&&$6$&$x^2 $&$- 8x$&&\\
&&&$-(6$&$x^2 $&$- 8x$&$)$&\\
\cline{4-7}\rule{0pt}{12pt}&&&&&$0$&&\\
\hline
\end{tabular}


\subsection{Potenzen}
\[ \boxed{a^n = a \cdot a^{n-1} \quad a^1 = a} \]
\[ \boxed{a^0 = 1 \quad \left(0^0\right)\text{ ist nicht definiert}} \]
\[ \boxed{a^{-1} = \frac{1}{a}} \]
\[ \boxed{a^{-n} = \frac{1}{a^n}} \]

\subsection{Potenzgesetze}
\[ \boxed{a^x \cdot a^y = a^{x+y}} \]
\[ \boxed{\frac{a^x}{a^y} = a^{x-y}} \]
\[ \boxed{(a^x)^y = a^{xy}} \]
\[ \boxed{a^x \cdot b^x = \left(ab\right)^x} \]
\[ \boxed{\frac{a^x}{b^x} = \left(\frac{a}{b}\right)^x} \]
\[ \boxed{a^{\frac{p}{q}} = \sqrt[q]{a^p}} \]

\subsection{Wurzeln}
\[ \boxed{\sqrt[n]{a^m} = a^{\frac{m}{n}} \quad \left(a>0; m, n \in \mathbb{N}; m \geq 1; n \geq 2\right)} \]
\[ \boxed{\sqrt[n]{1}=1 \quad \sqrt[n]{0}=0 \quad \sqrt[n]{a^n}=a \quad \left(a>0\right)} \]

\subsection{Wurzelgesetze}
\[ \boxed{\sqrt[n]{a}\cdot \sqrt[n]{b}=\sqrt[n]{a\cdot b}} \]
\[ \boxed{\frac{\sqrt[n]{a}}{\sqrt[n]{b}}=\sqrt[n]{\frac{a}{b}}} \]
\[ \boxed{\left(\sqrt[n]{a}\right)^k=\sqrt[n]{a^k}} \]
\[ \boxed{\sqrt[nk]{a^{mk}}=\sqrt[n]{a^m}} \]
\[ \boxed{\sqrt[n]{\sqrt[m]{a}}=\sqrt[m]{\sqrt[n]{a}}=\sqrt[mn]{a}} \]

\subsection{Logarithmengesetze}
\[ \boxed{y=\log_ax \Leftrightarrow a^y=x} \]
\[ \boxed{\log_a1=0 \quad \log_aa=1} \]
\[ \boxed{\lg a=\log_{10}a \quad \ln a = \log_ea} \]
\[ \boxed{a^{\log_ax}=x} \]
\[ \boxed{\log_a\left(a^x\right)=x} \]
% \subsubsection{Produkte}
\[ \boxed{\log_b\left(x \cdot y\right) = \log_b\left(x\right) + \log_b\left(y\right)} \]
% \subsubsection{Quotienten}
\[ \boxed{\log_b \left( \frac{x}{y} \right) = \log_b\left(x\right) - \log_b\left(y\right)} \]
% \subsubsection{Summen und Differenzen}
%\[ \boxed{\log_b\left(x \cdot y\right) = \log_b\left(x\right) + \log_b\left(y\right)} \]
\[ \boxed{\log_b\left(x + y\right) = \log_b\left(x\right) + \log_b\left(1 + \frac{y}{x}\right)} \]
% \subsubsection{Potenzen}
\[ \boxed{\log_b\left(x^r\right) = r \cdot \log_b\left(x\right)} \]
\[ \boxed{\log_ax=\frac{\lg x}{\lg a}=\frac{\ln x}{\ln a}} \]

\subsection{Trigonometrie}
$H$: Hypotenuse\\
$A$: Ankathete\\
$G$: Gegenkathete
\[ \boxed{\sin\alpha=\frac{G}{H}} \]
\[ \boxed{\cos\alpha=\frac{A}{H}} \]
\[ \boxed{\tan\alpha=\frac{G}{A}} \]
\[ \boxed{\cot\alpha=\frac{A}{G}} \]
\[ \boxed{\sin x = \sqrt{1-\cos^2x} = \sqrt{\frac{\tan^2x}{1+\tan^2x}}} \]
\[ \boxed{\cos x = \sqrt{1-\sin^2x} = \sqrt{\frac{1}{1+\tan^2x}}} \]
\[ \boxed{\tan x = \frac{\sin x}{\sqrt{1-\sin^2x}} = \frac{\sqrt{1-\cos^2x}}{\cos x} = \frac{\sin x}{\cos x}} \]
\[ \boxed{\sin^2 x + \cos^2 x = 1} \]

\subsection{Spezielle Werte der Winkelfunktionen}
\begin{tabular}{|l|c|c|c|c|c|}
\hline              & 0°$ = 0$ & 30°$ = \frac{\pi}{6}$ & 45°$ = \frac{\pi}{4}$ & 60°$ = \frac{\pi}{3}$ & 90°$ = \frac{\pi}{2}$ \\
\hline $f\left(x\right)=\sin x$ & $0$ & $\frac{1}{2}$ & $\frac{1}{2}\sqrt{2}$ & $\frac{1}{2}\sqrt{3}$ & $1$ \\
\hline $f\left(x\right)=\cos x$ & $1$ & $\frac{1}{2}\sqrt{3}$ & $\frac{1}{2}\sqrt{2}$ & $\frac{1}{2}$ & $0$ \\
\hline $f\left(x\right)=\tan x$ & $0$ & $\frac{1}{3}\sqrt{3}$ & $1$ & $\sqrt{3}$ & nicht def. \\
\hline $f\left(x\right)=\cot x$ & nicht def. & $\sqrt{3}$ & $1$ & $\frac{1}{3}\sqrt{3}$ & $0$ \\
\hline \end{tabular}

\subsection{Quadratische Gleichung}
\[ \boxed{f(x) = a \cdot x^2 + b \cdot x + c} \]
\[ \boxed{x_{1,2}=\frac{-b\pm\sqrt{b^2-4ac}}{2a}} \]

\subsection{Binomische Formeln}
Erste Binomische Formel: 
\[ \boxed{(a + b)^2 = a^2 + 2 \cdot a \cdot b + b^2} \]Zweite Binomische Formel: 
\[ \boxed{(a - b)^2 = a^2 - 2 \cdot a \cdot b + b^2} \]Dritte Binomische Formel: 
\[ \boxed{(a + b) \cdot (a - b) = a^2 - b^2} \]

\subsection{Verkettete Funktionen}
\[ \boxed{f(g(x)) = f \circ g} \]

\subsection{Grenzwerte}
\[ \boxed{\lim\limits_{x \to x_0}() = } \]
\[ \boxed{\lim\limits_{x \to x_0}() = } \]
\[ \boxed{\lim\limits_{x \to x_0}() = } \]
\[ \boxed{\lim\limits_{x \to x_0}() = } \]
\[ \boxed{\lim\limits_{x \to x_0}() = } \]
\[ \boxed{\lim\limits_{x \to x_0}() = } \]
\[ \boxed{\lim\limits_{x \to x_0}() = } \]
\[ \boxed{\lim\limits_{x \to x_0}() = } \]
\[ \boxed{\lim\limits_{x \to x_0}() = } \]
\[ \boxed{\lim\limits_{x \to x_0}() = } \]

\subsection{Berechnung von Grenzwerten}
Um Grenzwerte zu ermitteln muss der Ausdruck so angepasst werden, dass eindutig bestimmt werden kann, was sich daraus ergibt.
Dies wird durch Anwenden der zuvor aufgezeigten Rechnereglen erreicht.\\\\
Bsp.: $ \quad \quad \lim\limits_{n \rightarrow \infty} \left( \dfrac{\alpha \cdot n^2}{n^2-1} \right) $ \\\\
Um den Grenzwert zu ermitteln muss der Ausdruck erweitert werden und zwar so, dass
\begin{itemize}
\item der Term äquivalent bleibt
\item Variabeln des Indezes entfallen
\end{itemize}
Um die geforderten Bedingungen zu erfüllen wird der Term durch den reziproken Wert des Indezes mit der höchsten Potenz (im Zähler!) erweitert. In diesem Fall mit $\frac{1}{n^2}$. 
Eine weitere Regel besagt, dass Konstante Faktoren vorangenommen werden können.
Nun sieht es wie folgt aus:\\\\
\indent \indent \indent $ \alpha \lim\limits_{n \rightarrow \infty} 
    \left( \dfrac{\frac{1}{n^2} \cdot (n^2) }{ \frac{1}{n^2} \cdot (n^2 -1) } \right)  $\\\\\\
Vereinfacht man diesen Ausdruck durch elementare Algebra so erhält man: \\\\
\indent \indent \indent $ \alpha \lim\limits_{n \rightarrow \infty} \left( \dfrac{1}{1-
\smash{\underbrace{\frac{1}{n^2}}_0 } \vphantom{\dfrac{1}{n^2}} } \right) $\\\\\\
Der Audruck $\frac{1}{n^2}$ geht für $n \rightarrow \infty$ zu $0$. Daraus ergibt sich das folgende:\\\\\\
\indent \indent \indent $ \alpha \cdot 1 = \alpha $

