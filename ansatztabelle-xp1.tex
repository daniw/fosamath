% coding:utf-8

% Quelle: Supersaxo Mario

\subsubsection*{Ansatztabelle nach Supersaxo}
\def\arraystretch{1.5}
\noindent
\begin{longtable}{@{}l|l|l}
\hline
Störfunktion $s(x)$ & $\gamma$ & Standardansatz \\
\hline
$x$ 
	& $0$ 
	& $ax+b$ \\
$\xi x$ 
	& $0$ 
	& $ax+b$ \\
$x^2$ 
	& $0$ 
	& $ax^2+bx+c$ \\
$\xi x^2$ 
	& $0$ 
	& $ax^2+bx+c$ \\
\vdots & \vdots & \vdots \\
$sin(\omega x)$
	& $\omega j$
	& $asin(\omega x)+bcos(\omega x) $ \\
$x sin(\omega x)$ 
	& $\omega j$ 
	& $(ax+b)sin(\omega x)+(cx+d)cos(\omega x)$ \\
$x^2 sin(\omega x)$ 
	& $\omega j$ 
	& $(ax^2+bx+c)sin(\omega x)+(dx^2ex+f)cos(\omega x)$ \\
\vdots & \vdots & \vdots \\
$x^{\xi}sin(\omega x)+x^{\psi}cos(\omega x)$, $\xi>\psi$
	& $\omega$ 
	& $P_{\xi}(x)sin(\omega x) + P_{\xi}cos(\omega x)$ \\
$x^{\xi}sin(\omega x)+x^{\psi}cos(\omega x)$, $\xi<\psi$
	& $\omega$ 
	& $P_{\psi}(x)sin(\omega x) + P_{\psi}cos(\omega x)$ \\
\vdots & \vdots & \vdots \\
$e^{\omega x}$ 
	& $\omega$ 
	& $ae^{\omega x}$ \\ 
$\xi e^{\omega x}$ 
	& $\omega$ 
	& $ae^{\omega x}$ \\ 
$x+e^{\omega x}$ $\rightarrow s_1(x),s_2(x)$ 
	& $0,\omega$ 
	& $s_1\rightarrow ax+b$, $s_2\rightarrow ce^{\omega x}$ \\
\vdots & \vdots & \vdots \\
$x e^{\omega x}$ 
	& $\omega$ 
	& $(ax+b)e^{\omega x}$ \\
$(\xi+x)e^{\omega x}$ 
	& $\omega$ 
	& $(ax+b)e^{\omega x}$ \\
$xe^{\omega x}+\xi e^{\psi x} \rightarrow s_1(x), s_2(x)$ 
	& $\omega,\xi$ 
	& $s_1\rightarrow (ax+b)e^{\omega}, s_2\rightarrow ae^{\psi x}$ \\
\end{longtable}

\noindent
\textbf{Achtung:} Bei $e$-Funktionen muss überprüft werden ob der gewählte Ansatz 
eine Lösung der homogenen Gleichung darstellt. Falls dies zutrifft, muss der
Ansatz mit der unabängigen Variable erweitert werden. Dies wiederholt 
man so lange, bis es nicht mehr eine Lösung der homogenen Gleichung darstellt.
