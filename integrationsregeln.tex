% coding:utf-8

%----------------------------------------
%FOSAMATH, a LaTeX-Code for a mathematical summary for basic analysis
%Copyright (C) 2013, Daniel Winz, Ervin Mazlagic, Adrian Imboden, Philipp Langer

%This program is free software; you can redistribute it and/or
%modify it under the terms of the GNU General Public License
%as published by the Free Software Foundation; either version 2
%of the License, or (at your option) any later version.

%This program is distributed in the hope that it will be useful,
%but WITHOUT ANY WARRANTY; without even the implied warranty of
%MERCHANTABILITY or FITNESS FOR A PARTICULAR PURPOSE.  See the
%GNU General Public License for more details.
%----------------------------------------

% coding:utf-8
\section{Integrationsregeln}
\subsection{Partielles Integrieren}
\[ \boxed{\left. \int_a^b f'(x) \cdot g(x) dx = f(x) \cdot g(x) \right|_a^b - \int_a^b f(x) \cdot g'(x) dx} \]
Hier muss für $g(x)$ jener Ausdruck eingesetzt werden, welcher sich durch wiederholtes Ableiten eliminieren lässt.\\
Hinweis: Partielles integrieren verwendet man um einen Ausdruck zu eliminieren.
Dieser Vorgang kann so oft wie nötig wiederholt werden um etwas zu eliminieren (um $x^2$ zu eliminieren muss zwei mal hintereinander partiell integriert werden).
\subsection{Substitutionsregel}
\[ \boxed{\int_{g(a)}^{g(b)} f(x) dx = \int_{a}^{b} g'(x) \cdot f(g(x)) dx} \]

\subsection{Summen integrieren}
\[ \boxed{ \int \left( \sum_{K=a}^{b} (x)^K \right) dx= \sum_{K=a}^b \frac{(x)^{K+1}}{K+1} } \]

\section{Volumen}
\subsection{Rotationsvolumen}
\[ \boxed{V_x = \pi \int_{x_1}^{x_2} f(x)^2 dx} \quad \text{Rotation um die x-Achse}\]
\[ \boxed{V_y = \pi \int_{y_1}^{y_2} f^{-1}(y)^2 dy} \quad \text{Rotation um die y-Achse}\]

\section{Mittelwertsatz der Integralrechnung}
\[ \boxed{\int_{a}^{b} f(x) dx = f(\xi)(b-a)} \]

\section{Bogenlänge}
\[ \boxed{S = \int_{a}^{b} \sqrt{1 + y'(x)^2} dx} \]

\section{Bogenlänge für Kurven}
\[ \boxed{S = \int_{t_1}^{t_2}\sqrt{\dot{x}^2 + \dot{y}^2} dt \quad ,\gamma(t) = \left(\begin{matrix}x(t)\\y(t)\end{matrix}\right), t_1, t_2 \in I} \]

\section{Bogenlängen mit Polarkoordinaten}
\[ \boxed{S = \int_{\varphi_1}^{\varphi_2}\sqrt{\dot{r}^2(\varphi) + r^2(\varphi)}d\varphi} \]

\section{Mantelfläche}
\[ \boxed{A_{M_x} = 2 \pi \int_{x_1}^{x_2}f(x) \sqrt{1 + f'(x)^2} dx} \quad \text{Rotation um die x-Achse} \]
\[ \boxed{A_{M_y} = 2 \pi \int_{y_1}^{y_2}f^{-1}(y) \sqrt{1 + \frac{d}{dy}f^{-1}(y)^2} dy} \quad \text{Rotation um die y-Achse} \]

\section{Mantelfläche einer Kurve}
\[ \boxed{A_{M_x} = 2 \pi \int_{t_1}^{t_2} y(t) \sqrt{\dot{x}^2 + \dot{y}^2}dt} \quad \text{Bei Rotation um die x-Achse} \]
\[ \boxed{A_{M_y} = 2 \pi \int_{t_1}^{t_2} x(t) \sqrt{\dot{x}^2 + \dot{y}^2}dt} \quad \text{Bei Rotation um die y-Achse} \]

\section{Mantelfläche im polaren Koordinatensystem}
\[ \boxed{A_{M_x} = 2 \pi \int_{\varphi_1}^{\varphi_2}r(\varphi) \sin(\varphi)\sqrt{\dot{r}^2(\varphi) + r^2(\varphi)}} \quad \text{Bei Rotation um die x-Achse} \]
\[ \boxed{A_{M_y} = 2 \pi \int_{\varphi_1}^{\varphi_2}r(\varphi) \cos(\varphi)\sqrt{\dot{r}^2(\varphi) + r^2(\varphi)}} \quad \text{Bei Rotation um die y-Achse} \]
