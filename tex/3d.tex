% coding:utf-8

%----------------------------------------
%FOSAMATH, a LaTeX-Code for a mathematical summary for basic analysis
%Copyright (C) 2013, Daniel Winz, Ervin Mazlagic, Adrian Imboden, Philipp Langer

%This program is free software; you can redistribute it and/or
%modify it under the terms of the GNU General Public License
%as published by the Free Software Foundation; either version 2
%of the License, or (at your option) any later version.

%This program is distributed in the hope that it will be useful,
%but WITHOUT ANY WARRANTY; without even the implied warranty of
%MERCHANTABILITY or FITNESS FOR A PARTICULAR PURPOSE.  See the
%GNU General Public License for more details.
%----------------------------------------

% coding:utf-8
\section{Partielle Ableitung}
Die partielle Ableitung ist die Ableitung einer Funktion mit mehreren 
Variablen nach einer bestimmten Variable. Die anderen Variablen können dabei 
als konstant angesehen werden. 
\[ \boxed{\frac{\partial f(x,y)}{\partial x} = f_x(x,y) 
= \lim_{\Delta x \to 0} \frac{f(x + \Delta x, y) - f(x, y)}{\Delta x}} \]
\[ \boxed{\frac{\partial f(x,y)}{\partial y} = f_y(x,y) 
= \lim_{\Delta y \to 0} \frac{f(x, y + \Delta y) - f(x, y)}{\Delta y}} \]

\subsection{Mehrfache partielle Ableitung}
Wird eine Funktion mehrfach partiell abgeleitet, wird das wie folgt 
dargestellt: 
\newline
\begin{tabular}{@{}ll}
Doppelte partielle Ableitung nach $x$:          &$f_{xx}$ \\
Doppelte partielle Ableitung nach $y$:          &$f_{yy}$ \\
Partielle Ableitung erst nach $x$ und dann nach $y$: &$f_{xy}$ \\
Partielle Ableitung erst nach $y$ und dann nach $x$: &$f_{yx}$ \\
$\vdots$ & $\vdots$ 
\end{tabular} \\

\subsubsection{Satz von Schwarz}
Wenn die partiellen Ableitungen stetig sind, kann die Reihenfolge der 
Ableitungen beliebig vertauscht werden. 
\[ \boxed{\begin{array}{l}
f_{xy} = f_{yx} \\
f_{xyy} = f_{yxy} = f_{yyx} \\
f_{yxx} = f_{xyx} = f_{xxy} 
\end{array}} \]

\section{Kettenregel}
\[ \boxed{\frac{d f(x,y)}{d t} 
= \frac{\partial f}{\partial x} \cdot \frac{d x}{d t} 
+ \frac{\partial f}{\partial y} \cdot \frac{d y}{d t} 
= f_x \cdot \dot{x} + f_y \cdot \dot{y}} \]
\[ \boxed{\frac{d f(x,y,z)}{d t} 
= \frac{\partial f}{\partial x} \cdot \frac{d x}{d t} 
+ \frac{\partial f}{\partial y} \cdot \frac{d y}{d t} 
+ \frac{\partial f}{\partial z} \cdot \frac{d z}{d t} 
= f_x \cdot \dot{x} + f_y \cdot \dot{y} + f_z \cdot \dot{z}} \]

\section{Totales Differential}
\[ \boxed{d f 
= \frac{\partial f}{\partial x} dx + \frac{\partial f}{\partial y} dy 
= f_x dx + f_y dy} \]
\[ \boxed{d f = \frac{\partial f}{\partial x} dx 
+ \frac{\partial f}{\partial y} dy + \frac{\partial f}{\partial z} dz 
= f_x dx + f_y dy + f_z dz} \]

\section{Fehlerrechnung}
\[ \boxed{\Delta f(x,y) = \frac{\partial f(x,y)}{\partial x} \cdot \Delta x 
+ \frac{\partial f(x,y)}{\partial y} \cdot \Delta y 
= \underbrace{f_x \cdot \Delta x}_{a \cdot \Delta f} 
+ \underbrace{f_y \cdot \Delta y}_{(1 - a) \cdot \Delta f}} \]
$a$ ist dabei der Anteil, mit dem $x$ zum Gesamtfehler beiträgt. 
\[ \boxed{\begin{array}{rll}
a \cdot \Delta f        &= \frac{\partial f}{\partial x} \cdot \Delta x 
&= f_x \cdot \Delta x \\
(1 - a) \cdot \Delta f  &= \frac{\partial f}{\partial y} \cdot \Delta y 
&= f_y \cdot \Delta y \\
\end{array}} \]

\newpage

\section{Kurvendiskussion im dreidimensionalen Raum}

\subsection{Extremwert ohne Nebenbedingung}
\[ \boxed{f_x \stackrel{!}{=} 0 \land f_y \stackrel{!}{=} 0} \]
\[ \boxed{\begin{array}{@{}l}
\Delta = f_{xx}(x_0, y_0) \cdot f_{yy}(x_0, y_0) - {f_{xy}}^2(x_0, y_0) \\
\begin{array}{ll}
\Delta < 0 \quad                    & \rightarrow \text{Sattelpunkt} \\
\Delta > 0 \land f_{xx} < 0 \quad   & \rightarrow \text{Relatives Maximum} \\
\Delta > 0 \land f_{xx} > 0 \quad   & \rightarrow \text{Relatives Minimum} \\
\Delta = 0                  \quad   & \rightarrow \text{kein Entscheid möglich}
\end{array}\end{array}} \]

\subsection{Extremwert mit Nebenbedingung}
\[ \boxed{\begin{array}{@{}ll}
f(x,y) & \text{Optimierungsfunktion} \\
g(x,y) = 0 & \text{Nebenbedingung in impliziter Form} \\
\end{array}} \]
\[ \boxed{\Lagr(x,y,\lambda) = f(x,y) + \lambda \cdot g(x,y)} \]
\[ \boxed{\begin{array}{l}
\Lagr_x = f_x(x,y) + \lambda \cdot g_x(x,y) \stackrel{!}{=} 0 \\
\Lagr_y = f_y(x,y) + \lambda \cdot g_y(x,y) \stackrel{!}{=} 0 \\
\Lagr_\lambda = g(x,y) \stackrel{!}{=} 0 \\
\end{array}} \]
$\lambda$ ist der Lagrangesche Multiplikator. 

\subsection{Extremwert mit zwei Nebenbedingungen}
\[ \boxed{\begin{array}{@{}ll}
f(x,y) & \text{Optimierungsfunktion} \\
g_1(x,y) = 0 & \text{Nebenbedingung 1 in impliziter Form} \\
g_2(x,y) = 0 & \text{Nebenbedingung 2 in impliziter Form} \\
\end{array}} \]
\[ \boxed{\Lagr(x,y,\lambda,\mu) = f(x,y) + \lambda \cdot g_1(x,y) + \mu \cdot g_2(x,y)} \]
\[ \boxed{\begin{array}{l}
\Lagr_x = f_x(x,y) + \lambda \cdot g_{1_x}(x,y) + \mu \cdot g_{2_x}(x,y) 
\stackrel{!}{=} 0 \\
\Lagr_y = f_y(x,y) + \lambda \cdot g_{1_y}(x,y) + \mu \cdot g_{2_x}(x,y) 
\stackrel{!}{=} 0 \\
\Lagr_\lambda = g_1(x,y) \stackrel{!}{=} 0 \\
\Lagr_\mu = g_2(x,y) \stackrel{!}{=} 0 \\
\end{array}} \]
$\lambda$ und $\mu$ sind die Lagrangeschen Multiplikatoren. 

\subsection{Extremwert mit mehreren Nebenbedingungen}
\[ \boxed{\Lagr(x_1, \dots, x_n, \lambda_1, \dots, \lambda_m) 
= f(x_1, \dots, x_n) 
+ \sum_{i=1}^{m} \lambda_i \cdot g_i(x_1, \dots, x_n)} \]
$\lambda_i$ sind die Lagrangeschen Multiplikatoren. 

\section{Doppelintegral}
\[ \boxed{\iint\limits_{(A)} f(x,y) ~ dA 
= \underbrace{\int_{x=a}^{b} 
\underbrace{\int_{y=f_u(x)}^{f_o(x)} f(x,y) ~ dy}
_{\text{Inneres Integral}} ~ dx}
_{\text{Äusseres Integral}}} \]

\subsection{Doppelintegral in Polarkoordinaten}
\[ \boxed{\iint\limits_{(A)} f(x,y) ~ dA 
= \underbrace{\int_{\varphi=\varphi_1}^{\varphi_2} 
\underbrace{\int_{r=r_i(\varphi)}^{r_a(\varphi)} f(r \cdot \cos(\varphi), 
r \cdot \sin(\varphi)) \cdot r ~ dr}
_{\text{Inneres Integral}} ~ d\varphi}
_{\text{Äusseres Integral}}} \]

\section{Dreifachintegral}
\[ \boxed{\iiint\limits_{(V)} f(x,y,z) ~ dV 
= \underbrace{\int_{x=a}^{b} 
\underbrace{\int_{y=f_u(x)}^{f_o(x)} 
\underbrace{\int_{z=z_u(x,y)}^{z_o(x,y)} f(x,y,z) ~ dz}
_{\text{1. Integration}} ~ dy}
_{\text{2. Integration}} ~ dx}
_{\text{3. Integration}}} \]

\subsection{Dreifachintegral in Zylinderkoordinaten}
\[ \boxed{\iiint\limits_{(V)} f(x,y,z) ~ dV 
= \iiint\limits_{(V)} f(r \cdot \cos(\varphi),r \cdot \sin(\varphi), z) 
\cdot r ~ dz ~ dr ~ d\varphi} \]

\subsection{Dreifachintegral in Kugelkoordinaten}
\[ \boxed{\iiint\limits_{(V)} f(x,y,z) ~ dV = \iiint\limits_{(V)} 
f\left(\left(\begin{array}{@{}l@{}}r \cdot \sin(\vartheta) \cdot \cos(\varphi) \\
r \cdot \sin(\vartheta) \cdot \sin(\varphi) \\ 
r \cdot \cos(\vartheta)\end{array}\right)\right) 
\cdot r^2 \cdot \sin(\vartheta) ~ dr ~ d\vartheta ~ d\varphi} \]

\section{Anwendungen}

\subsection{Volumen}
\[ \boxed{V = \iiint\limits_{(V)} 1 ~ dV} \]

\subsection{Masse}
\[ \boxed{m = \iiint\limits_{(V)} \rho(x,y,z) ~ dV} \]
Ist $\rho$ konstant, kann $\rho$ vor das Integral verschoben werden. 

\subsection{Flächenträgheitsmoment}
Kartesische Koordinaten
\[ \boxed{\begin{array}{@{}l}
\displaystyle I_x = \iint\limits_{(A)} y^2 ~ dA \\
\displaystyle I_y = \iint\limits_{(A)} x^2 ~ dA \\
\displaystyle I_p = \iint\limits_{(A)} (x^2 + y^2) ~ dA \\
\end{array}} \]
Polarkoordinaten
\[ \boxed{\begin{array}{@{}l}
\displaystyle I_x = \iint\limits_{(A)} r^3 \cdot \sin^2(\varphi) ~ dA \\
\displaystyle I_y = \iint\limits_{(A)} r^3 \cdot \cos^2(\varphi) ~ dA \\
\displaystyle I_p = \iint\limits_{(A)} r^3 ~ dA \\
\end{array}} \]

\subsection{Massenträgheitsmoment}
\[ \boxed{I = \rho \cdot \iiint\limits_{(V)} r^2 ~ dV 
= \rho \cdot \iiint\limits_{(V)} (x^2 + y^2) ~ dV} \]
Rotationskörper
\[ \boxed{I_z = \rho \cdot \iiint\limits_{(V)} r^3 ~ dz ~ dr ~ d\varphi} \]

\subsection{Schwerpunkt}
Kartesische Koordinaten
\[ \boxed{\begin{array}{@{}l}
\displaystyle x_s = \frac{1}{A} \iint\limits_{(A)} x ~ dA \\
\displaystyle y_s = \frac{1}{A} \iint\limits_{(A)} y ~ dA \\
\end{array}} \]
\[ \boxed{\begin{array}{@{}l}
\displaystyle x_s = \frac{1}{V} \iiint\limits_{(V)} x ~ dV \\
\displaystyle y_s = \frac{1}{V} \iiint\limits_{(V)} y ~ dV \\
\displaystyle z_s = \frac{1}{V} \iiint\limits_{(V)} z ~ dV \\
\end{array}} \]
Polarkoordinaten
\[ \boxed{\begin{array}{@{}l}
\displaystyle x_s 
= \frac{1}{A} \iint\limits_{(A)} r^2 \cdot \cos(\varphi) ~ dA \\
\displaystyle y_s 
= \frac{1}{A} \iint\limits_{(A)} r^2 \cdot \sin(\varphi) ~ dA \\
\end{array}} \]
Rotationskörper
\[ \boxed{\begin{array}{@{}l}
\displaystyle x_s = 0 \\
\displaystyle y_s = 0 \\
\displaystyle z_s = \frac{1}{V} \iiint\limits_{(V)} z \cdot r ~ dV \\
\end{array}} \]

\section{Potentialfeld}
\[ \boxed{\vec{v} = \grad \phi =
\left(\begin{array}{@{}l@{}} 
\frac{\partial \phi}{\partial x} \\
\frac{\partial \phi}{\partial y} \\
\end{array}\right) \qquad \text{$\phi$ ist das Potentialfeld}} \]
\[ \boxed{\vec{v} = \grad \phi =
\left(\begin{array}{@{}l@{}} 
\frac{\partial \phi}{\partial x} \\
\frac{\partial \phi}{\partial y} \\
\frac{\partial \phi}{\partial z} \\
\end{array}\right) \qquad \text{$\phi$ ist das Potentialfeld}} \]

\section{Gradient}\label{sec:gradient}
Der Gradient zeigt in die Richtung des steilsten Anstieges. Seine Länge 
entspricht der Steigung. 
\[ \boxed{\grad \phi = \frac{\partial \phi}{\partial x} \cdot \vec{e}_x 
+ \frac{\partial \phi}{\partial y} \cdot \vec{e}_y 
= \left(\begin{array}{@{}l@{}} 
\frac{\partial \phi}{\partial x} \\
\frac{\partial \phi}{\partial y} \\
\end{array}\right)} \]
\[ \boxed{\grad \phi = \frac{\partial \phi}{\partial x} \cdot \vec{e}_x 
+ \frac{\partial \phi}{\partial y} \cdot \vec{e}_y 
+ \frac{\partial \phi}{\partial z} \cdot \vec{e}_z 
= \left(\begin{array}{@{}l@{}} 
\frac{\partial \phi}{\partial x} \\
\frac{\partial \phi}{\partial y} \\
\frac{\partial \phi}{\partial z} \\
\end{array}\right)} \]

\subsection{Richtungsableitung}
Die Richtungsableitung ist die Steigung in Richtung von $\vec{a}$. 
\[ \boxed{\frac{\partial \phi}{\partial \vec{a}} 
= (\grad \phi) \cdot \vec{e}_a 
= (\grad \phi) \cdot \frac{\vec{a}}{||\vec{a}||}} \]

\section{Rotation}
\[ \boxed{\rot \vec{F} = \left( \begin{array}{@{}l@{}} 
\frac{\partial F_z}{\partial y} - \frac{\partial F_y}{\partial z} \\
\frac{\partial F_x}{\partial z} - \frac{\partial F_z}{\partial x} \\
\frac{\partial F_y}{\partial x} - \frac{\partial F_x}{\partial y} \\
\end{array}\right)} \]
Wenn $\rot\vec{F}=0$, ist das Vektorfeld $\vec{F}$ konservativ. \\
$\vec{F}$ kann dann ein Potentialfeld haben.

\section{Divergenz}
\[ \boxed{\diver \vec{F} = \frac{\partial F_x}{\partial x} 
+ \frac{\partial F_y}{\partial y} + \frac{\partial F_z}{\partial z}} \]
$F_x, F_y, F_z$: Skalare Komponenten des Vektorfeldes $\vec{F}(x,y,z)$\\
$\diver > 0$: Quelle \\
$\diver < 0$: Senke \\
$\diver = 0$: Quellenfrei

\section{Linienintegral}
\[ \boxed{\int\limits_{C} \vec{F} \cdot d\vec{r} 
= \int\limits_{t_1}^{t_2} \left(\vec{F}(t) \cdot \dot{\vec{r}}\right) ~ dt} \]
Existiert zum Vektorfeld $\vec{F}$ ein Potentialfeld $\phi$, so kann das 
Linienintegral mittels Anfangs- und Endpunkt des Weges berechnet werden. 
\[ \boxed{\int\limits_{C} \vec{F} \cdot d \vec{r} = \phi(B) - \phi(A)} 
\qquad \vec{r} = \overrightarrow{AB} \]

\subsection{Bogenlänge}
\[ \boxed{S = \int\limits_{t_1}^{t_2} \left(|| \dot{\vec{r}} ||\right) ~dt} \]

\section{Flussintegral}
\[ \boxed{\iint\limits_{(A)} \vec{F} \cdot d\vec{A} 
= \iint\limits_{(A)} (\vec{F} \cdot \vec{N}) ~ dA} \]

\subsection{Normalenvektor}
Der Normalenvektor hat die Länge $1$ und steht senkrecht auf einer Fläche. \\
Plane Fläche: 
\[ \boxed{\vec{N} 
= \frac{\overrightarrow{AB} \times \overrightarrow{AC}}
{||\overrightarrow{AB} \times \overrightarrow{AC}||}} 
\qquad \text{$A, B, C$: Punkte auf der Fläche} \]
Beliebige Fläche: 
\[ \boxed{\vec{N} 
= \frac{\frac{\partial \phi}{\partial \vec{x}} 
\times \frac{\partial \phi}{\partial \vec{y}}}
{||\frac{\partial \phi}{\partial \vec{x}} 
\times \frac{\partial \phi}{\partial \vec{y}}||}} \]
