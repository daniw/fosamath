\section{Bewegungen}

\subsection{Geschwindigkeitsvektor}
Ist eine Bewegung gegeben durch einen Ortsvektor $\vec{r}(t)$, dann kann 
die Bewegung eines Punktes $P(t)$ dieser Kurve mittels eines 
Geschwindigkeitsvektors ausgedrückt werden in der Art
\[ 
	\vec{v}(t) = \dot{\vec{r}}(t) 
	= \begin{pmatrix} 
		\dot{x}(t) \\ 
		\dot{y}(t) \\ 
		\dot{z}(t) 
	\end{pmatrix} 
\]
Dieser Vektor ist somit immer auch ein Tangentialvektor $\vec{t}$ an die 
Kurve im Punkt $P(t)$. Für den skalaren Wert der Geschwindigkeit selbst wird 
die Norm des Geschwindigkeitsvektors genommen.
\[  
	v(t) 
	= \left\lVert \dot{\vec{r}}(t) \right\rVert 
	= \left\lVert \begin{pmatrix} 
		\dot{x}(t) \\ 
		\dot{y}(t) \\ 
		\dot{z}(t) 
	\end{pmatrix} \right\rVert
	= \sqrt{\dot{x}^2(t) + \dot{y}^2(t) + \dot{z}^2(t)}
\]

\subsection{Beschleunigungsvektor}
Analog zur Geschwindigkeit ist die Beschleunigung die zweite Ableitung
des Weges einer Kurve $\vec{r}(t)$.
\[ 
	\vec{a}(t) 
	= \dot{\vec{v}}(t) 
	= \ddot{\vec{r}}(t) 
	= \begin{pmatrix} 
		\ddot{x}(t) \\ 
		\ddot{y}(t) \\ 
		\ddot{z}(t) 
	\end{pmatrix} 
\]
Genauso kann wie bei der Geschwindigkeit auch bei der Beschleunigung die
Norm benutzt werden um den skalaren Wert der Beschleunigung zu erhalten
\[  
	a(t) 
	= \left\lVert \ddot{\vec{r}}(t) \right\rVert 
	= \left\lVert \begin{pmatrix} 
		\ddot{x}(t) \\ 
		\ddot{y}(t) \\ 
		\ddot{z}(t) 
	\end{pmatrix} \right\rVert
	= \sqrt{\ddot{x}^2(t) + \ddot{y}^2(t) + \ddot{z}^2(t)}
\]

\subsection{Bogenlänge}
Der zurückgelegte Weg einer Kurve $\vec{r}(t)$ von $A(t_1)$ zu $B(t_2)$ kann
durch das Integral von der Norm der Geschwindigkeit ermittelt werden
\[  
	s = \int\limits_{t_1}^{t_2} \left\lVert \vec{v}(t) \right\lVert \, dt
	= \int\limits_{t_1}^{t_2} \left\lVert \dot{\vec{r}}(t) \right\lVert \, dt
	= \int\limits_{t_1}^{t_2} 
		\sqrt{\dot{x}^2(t) + \dot{y}^2(t) + \dot{z}^2(t)} \,dt
\]
