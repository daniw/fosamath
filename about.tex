% coding:utf-8

%----------------------------------------
%FOSAMATH, a LaTeX-Code for a mathematical summary for basic analysis
%Copyright (C) 2013, Daniel Winz, Ervin Mazlagic, Adrian Imboden, Philipp Langer

%This program is free software; you can redistribute it and/or
%modify it under the terms of the GNU General Public License
%as published by the Free Software Foundation; either version 2
%of the License, or (at your option) any later version.

%This program is distributed in the hope that it will be useful,
%but WITHOUT ANY WARRANTY; without even the implied warranty of
%MERCHANTABILITY or FITNESS FOR A PARTICULAR PURPOSE.  See the
%GNU General Public License for more details.
%----------------------------------------

\chapter*{Über diese Arbeit}
Dies ist das Ergebnis einer Zusammenarbeit auf Basis freier Texte erstellt von 
Studierenden der Fachhochschule Luzern und ist unter der GPLv2 lizenziert. 
Der \TeX - bzw. \LaTeX -Code ist auf \url{github.com/fosa/fosamath} 
hinterlegt. Eine aktuelle PDF-Ausgabe steht auf \url{fosa.adinox.ch} zum 
Download bereit.

Die Inhalte dieser Formelsammlung sind für das Modul Mathematik Grundlagen 
und den Mathematikteil von Mathematik-Physik 2 zugeschnitten. 
%
\iftiboth
	Zudem sind in dieser Formelsamlung Tipps und Hinweise für die Bedienung 
    des TI-89 und des TI-Nspire CAS enthalten. 
	\else
	\ifti
		Zudem sind in dieser Formelsammlung Tipps und Hinweise für die 
        Bedienung des TI-89 enthalten. 
	\fi
	\ifnspire
		Zudem sind in dieser Formelsammlung Tipps und Hinweise für die 
        Bedienung des TI-Nspire CAS enthalten. 
	\fi
\fi

Allfällige Fehler und Kommentare werden von den Autoren gerne entgegengenommen
(\href{mailto:nino.ninux@gmail.com}{\nolinkurl{nino.ninux@gmail.com}} oder 
\href{mailto:daniel.winz@stud.hslu.ch}{\nolinkurl{daniel.winz@stud.hslu.ch}}).

\section*{Danksagung}
An dieser Stelle möchten wir allen danken, die uns bei diesem Projekt 
unterstützt haben.
Einerseits sind dies alle Contributors auf dem Github-Repository 
\verb!fosamath! und jene Studenten die uns Rückmeldungen gegeben haben.
Ein spezieller Dank geht dabei an unsere Dozenten Mario Amrein und Peter 
Scheiblechner, welche uns eine unschätzbare Hilfe waren.
Sie haben uns nicht nur bei der Erarbeitung und Überprüfung der Formelsammlung 
geholfen, sondern uns auch mit Ihren Vorlesungen für die Mathematik begeistert 
und motiviert.


