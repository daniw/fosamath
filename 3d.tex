% coding:utf-8

%----------------------------------------
%FOSAMATH, a LaTeX-Code for a mathematical summary for basic analysis
%Copyright (C) 2013, Daniel Winz, Ervin Mazlagic, Adrian Imboden, Philipp Langer

%This program is free software; you can redistribute it and/or
%modify it under the terms of the GNU General Public License
%as published by the Free Software Foundation; either version 2
%of the License, or (at your option) any later version.

%This program is distributed in the hope that it will be useful,
%but WITHOUT ANY WARRANTY; without even the implied warranty of
%MERCHANTABILITY or FITNESS FOR A PARTICULAR PURPOSE.  See the
%GNU General Public License for more details.
%----------------------------------------

% coding:utf-8
\section{Partielle Ableitung}
Die partielle Ableitung ist die Ableitung einer Funktion mit mehreren 
Variablen nach einer bestimmten Variable. Die anderen Variablen können dabei 
als konstant angesehen werden. 
\[ \boxed{\frac{\partial f(x,y)}{\partial x} = f_x(x,y) 
= \lim_{\Delta x \to 0} \frac{f(x + \Delta x, y) - f(x, y)}{\Delta x}} \]
\[ \boxed{\frac{\partial f(x,y)}{\partial y} = f_y(x,y) 
= \lim_{\Delta y \to 0} \frac{f(x, y + \Delta y) - f(x, y)}{\Delta y}} \]

\subsection{Mehrfache partielle Ableitung}
Wird eine Funktion mehrfach partiell abgeleitet, wird das wie folgt 
dargestellt: 
\newline
\begin{tabular}{@{}ll}
Doppelte partielle Ableitung nach $x$:          &$f_{xx}$ \\
Doppelte partielle Ableitung nach $y$:          &$f_{yy}$ \\
Partielle Ableitung erst nach $x$ und dann nach $y$: &$f_{xy}$ \\
Partielle Ableitung erst nach $y$ und dann nach $x$: &$f_{yx}$ \\
$\vdots$ & $\vdots$ 
\end{tabular} \\

\subsubsection{Satz von Schwarz}
Wenn die partiellen Ableitungen stetig sind, kann die Reihenfolge der 
Ableitungen beliebig vertauscht werden. 
\[ \boxed{\begin{array}{l}
f_{xy} = f_{yx} \\
f_{xyy} = f_{yxy} = f_{yyx} \\
f_{yxx} = f_{xyx} = f_{xxy} 
\end{array}} \]

\section{Kettenregel}
\[ \boxed{\frac{d f(x,y)}{d t} 
= \frac{\partial f}{\partial x} \cdot \frac{d x}{d t} 
+ \frac{\partial f}{\partial y} \cdot \frac{d y}{d t} 
= f_x \cdot \dot{x} + f_y \cdot \dot{y}} \]
\[ \boxed{\frac{d f(x,y,z)}{d t} 
= \frac{\partial f}{\partial x} \cdot \frac{d x}{d t} 
+ \frac{\partial f}{\partial y} \cdot \frac{d y}{d t} 
+ \frac{\partial f}{\partial z} \cdot \frac{d z}{d t} 
= f_x \cdot \dot{x} + f_y \cdot \dot{y} + f_z \cdot \dot{z}} \]

\section{Totales Differential}
\[ \boxed{d f 
= \frac{\partial f}{\partial x} dx + \frac{\partial f}{\partial y} dy 
= f_x dx + f_y dy} \]
\[ \boxed{d f = \frac{\partial f}{\partial x} dx 
+ \frac{\partial f}{\partial y} dy + \frac{\partial f}{\partial z} dz 
= f_x dx + f_y dy + f_z dz} \]

\section{Fehlerrechnung}
\[ \boxed{\Delta f(x,y) = \frac{\partial f(x,y)}{\partial x} \cdot \Delta x 
+ \frac{\partial f(x,y)}{\partial y} \cdot \Delta y 
= \underbrace{f_x \cdot \Delta x}_{a \cdot \Delta f} 
+ \underbrace{f_y \cdot \Delta y}_{(1 - a) \cdot \Delta f}} \]
$a$ ist dabei der Anteil, mit dem $x$ zum Gesamtfehler beiträgt. 
\[ \boxed{\begin{array}{rll}
a \cdot \Delta f        &= \frac{\partial f}{\partial x} \cdot \Delta x 
&= f_x \cdot \Delta x \\
(1 - a) \cdot \Delta f  &= \frac{\partial f}{\partial y} \cdot \Delta y 
&= f_y \cdot \Delta y \\
\end{array}} \]

\section{Kurvendiskussion im dreidimensionalen Raum}

\subsection{Extremwert ohne Nebenbedingung}
\[ \boxed{f_x = 0 \land f_y = 0} \]
\[ \boxed{\begin{array}{@{}l}
\Delta = f_{xx}(x_0, y_0) \cdot f_{yy}(x_0, y_0) - {f_{xy}}^2(x_0, y_0) \\
\begin{array}{ll}
\Delta < 0 \quad                    & \rightarrow \text{Sattelpunkt} \\
\Delta > 0 \land f_{xx} < 0 \quad   & \rightarrow \text{Relatives Maximum} \\
\Delta > 0 \land f_{xx} > 0 \quad   & \rightarrow \text{Relatives Minimum} \\
\Delta = 0                  \quad   & \rightarrow \text{kein Entscheid möglich}
\end{array}\end{array}} \]

\subsection{Extremwert mit Nebenbedingung}
\[ \boxed{\begin{array}{@{}ll}
f(x,y) & \text{Optimierungsfunktion} \\
g(x,y) = 0 & \text{Nebenbedingung in impliziter Form} \\
\end{array}} \]
\[ \boxed{\Lagr(x,y,\lambda) = f(x,y) + \lambda \cdot g(x,y)} \]
\[ \boxed{\begin{array}{l}
\Lagr_x = f_x(x,y) + \lambda \cdot g_x(x,y) = 0 \\
\Lagr_y = f_y(x,y) + \lambda \cdot g_y(x,y) = 0 \\
\Lagr_\lambda = g(x,y) = 0 \\
\end{array}} \]
$\lambda$ ist der Lagrangesche Multiplikator. 

\subsection{Extremwert mit zwei Nebenbedingungen}
\[ \boxed{\begin{array}{@{}ll}
f(x,y) & \text{Optimierungsfunktion} \\
g_1(x,y) = 0 & \text{Nebenbedingung 1 in impliziter Form} \\
g_2(x,y) = 0 & \text{Nebenbedingung 2 in impliziter Form} \\
\end{array}} \]
\[ \boxed{\Lagr(x,y,\lambda) = f(x,y) + \lambda \cdot g_1(x,y) + \mu \cdot g_2(x,y)} \]
\[ \boxed{\begin{array}{l}
\Lagr_x = f_x(x,y) + \lambda \cdot g_{1_x}(x,y) + \mu \cdot g_{2_x}(x,y) = 0 \\
\Lagr_y = f_y(x,y) + \lambda \cdot g_{1_y}(x,y) + \mu \cdot g_{2_x}(x,y) = 0 \\
\Lagr_\lambda = g_1(x,y) = 0 \\
\Lagr_\mu = g_2(x,y) = 0 \\
\end{array}} \]
$\lambda$ und $\mu$ sind die Lagrangeschen Multiplikatoren. 

\subsection{Extremwert mit mehreren Nebenbedingungen}
\[ \boxed{\Lagr(x_1, \dots, x_n, \lambda_1, \dots, \lambda_m) 
= f(x_1, \dots, x_n) 
+ \sum_{i=1}^{m} \lambda_i \cdot g_i(x_1, \dots, x_n)} \]

\section{Doppelintegral}
\section{Dreifachintegral}
\section{Anwendungen}
\subsection{Massenträgheitsmoment}
\subsection{Flächenträgheitsmoment}
\subsection{Schwerpunkt}
\section{Potentialfeld}
\section{Gradient}
\section{Rotation}
\section{Divergenz}
\section{Langrangescher Multiplikator}
\section{Linienintegral}
\section{Flussintegral}
