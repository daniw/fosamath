% coding:utf-8
\section{Kurvendiskussion}

\subsection{Tangentengleichnung}
\[ \boxed{T(x) = f'(x_0)(x - x_0) + f(x_0)} \]

\subsection{Steigen und Fallen}

\[ \boxed{ \begin{matrix}
f'(x) > 0 \text{ auf } I & \Rightarrow  & f \text{ ist monoton wachsend in $I$} \\
f'(x) < 0 \text{ auf } I & \Rightarrow  & f \text{ ist monoton fallend in $I$}
\end{matrix} } \]

\noindent
$I$ entspricht einem Intervall! Dies bedeutet, ist $f'(x)$ über den gesamten Bereich immer $>0$ so ist $f$ monoton wachsend.
Ist $f'(x)$ über den gesamten Bereich $<0$ so ist sie monoton fallend.

\subsection{Krümmungsverhalten}

\[ \boxed{ \begin{matrix}
f''(x) > 0 \text{ auf } I & \Rightarrow  & \text{ Kurve ist konvex } \\
f''(x) < 0 \text{ auf } I & \Rightarrow  & \text{ Kurve ist konkav }
\end{matrix} } \]

\subsection{Extremum}
Ein Extremum ist ein Punkt, zu welchem die Ableitung $0$ ergibt.
Solch ein Extremum kann ein Maximum oder Minimum sein.
Zusätzlich ist zu definieren ob es sich um ein lokales oder globales Extremum handelt.

\[ \boxed{ \begin{matrix}
f'(x_0) = 0 \land f''(x_0) < 0 & \Rightarrow & \text{lokales Maximum in $x_0$} \\
f'(x_0) = 0 \land f''(x_0) > 0 & \Rightarrow & \text{lokales Minimum in $x_0$} 
\end{matrix} } \]

\subsection{Wendepukt}
Als Wendepukt bezeichnet man jene Stelle, an welcher die Krümmung der Funktion wechselt (konkav zu konvex und umgekehrt).
Im Wendepunkt ist die Steigung jeweils von beiden Seiten aus betrachtet (d.h aus $x_0 > 0$ und $x_0<0$) extremal.

\[ \boxed{ \begin{matrix}
f''(x_0) = 0 \land f'''(x_0) \neq 0 & \Rightarrow & \text{Wendepunkt in $x_0$}
\end{matrix} } \]

\subsection{Sattelpunkt}
Ein Sattelpunkt ist ein Wendepunkt mit horizontaler Wendetangente.

\[ \boxed{ \begin{matrix}
f'(x_0) =  f''(x_0) = 0 \land f'''(x_0) \neq 0 & \Rightarrow & \text{Sattelpunkt in $x_0$}
\end{matrix} } \]